\documentclass{article}

\usepackage{tabularx}
\usepackage{booktabs}
\usepackage{hyperref}  

\title{Problem Statement and Goals\\\progname}

\author{\authname}

\date{}

%% Comments

\usepackage{color}

\newif\ifcomments\commentstrue %displays comments
%\newif\ifcomments\commentsfalse %so that comments do not display

\ifcomments
\newcommand{\authornote}[3]{\textcolor{#1}{[#3 ---#2]}}
\newcommand{\todo}[1]{\textcolor{red}{[TODO: #1]}}
\else
\newcommand{\authornote}[3]{}
\newcommand{\todo}[1]{}
\fi

\newcommand{\wss}[1]{\authornote{magenta}{SS}{#1}} 
\newcommand{\plt}[1]{\authornote{cyan}{TPLT}{#1}} %For explanation of the template
\newcommand{\an}[1]{\authornote{cyan}{Author}{#1}}

%% Common Parts

\newcommand{\progname}{Software Engineering} % PUT YOUR PROGRAM NAME HERE
\newcommand{\authname}{Team 13, Speech Buddies
\\ Mazen Youssef
\\ Rawan Mahdi
\\ Luna Aljammal
\\ Kelvin Yu} % AUTHOR NAMES                  

\usepackage{hyperref}
    \hypersetup{colorlinks=true, linkcolor=blue, citecolor=blue, filecolor=blue,
                urlcolor=blue, unicode=false}
    \urlstyle{same}
                                


\begin{document}

\maketitle

\begin{table}[hp]
\caption{Revision History} \label{TblRevisionHistory}
\begin{tabularx}{\textwidth}{llX}
\toprule
\textbf{Date} & \textbf{Developer(s)} & \textbf{Change}\\
\midrule
Sept 18 & Mazen & Drafted \& Reviewed the Problem Statement/Goals\\
Sept 22 & Mazen & Made Changes based on team's Feedback and filled Appendix\\
Sept 28 & Rawan & Added Citations to problem statement based on peer review\\
Oct 30 & Mazen & Implemented TA Feedback\\
\bottomrule
\end{tabularx}
\end{table}

\section{Problem Statement}
This section outlines the problem that our project aims to solve, detailing the inputs and outputs of the system, identifying key stakeholders, and describing the environment in which the solution will operate.
\subsection{Problem}
Individuals with speech impairments, such as those caused by neurological conditions (e.g., Parkinson's disease, stroke, or cerebral palsy), often struggle to interact with modern digital devices \cite{nidcd2024assistive}. Though there have been recent advancements in state-of-the-art Automatic Speech Recognition (ASR) models, widely used models like Whisper and wav2vec show significantly reduced performance on atypical speech \cite{pokel2025variational}. The lack of accurate tools for this population limits their independence, restricts accessibility, and creates barriers to everyday tasks such as sending messages, browsing the internet, or operating applications. A solution is needed that can reliably interpret impaired speech and seamlessly integrate with existing devices to empower users with greater autonomy.

\subsection{Inputs and Outputs}
  Inputs:
    \begin{itemize}
      \item At a high level, the system will take speech audio input from users with impaired or slurred speech, captured through a microphone.
    \end{itemize}
  Processing:
    \begin{itemize}
      \item The input is then processed by a custom-trained machine learning model designed to handle atypical speech patterns.
    \end{itemize}
  Outputs:
    \begin{itemize}
      \item The system will generate text output representing the user's intended words.
      \item This text is further mapped into commands or actions that can be executed on a phone or computer interface, allowing the user to control applications, send messages, or perform other digital tasks without requiring standard speech clarity.

    \end{itemize}
    By bridging the gap between impaired speech and device interaction, the system transforms inaccessible voice commands into actionable outcomes.

\subsection{Stakeholders}
The primary stakeholders are individuals with speech impairments or dysarthic speech. These users directly benefit from improved accessibility and independence in controlling digital devices.

Secondary stakeholders include:
\begin{itemize}
  \item Caregivers and family members, who gain relief knowing their loved ones can communicate and operate devices more independently.
  \item Healthcare professionals and speech therapists, who can recommend or integrate this technology into treatment plans to support rehabilitation and daily functioning.
  \item Accessibility advocates and organizations, who seek inclusive technology solutions to close the digital accessibility gap.
  \item Software developers and device manufacturers, who may integrate this system into their platforms to expand accessibility features.
  \item Researchers in speech recognition and assistive technologies, who can build upon this project to further improve accessibility solutions.
\end{itemize}

\subsection{Environment}
The system is expected to operate on a personal computer that is equipped with a standard microphone for capturing speech input. It must function in everyday environments where background noise may be present, such as at home, in clinics, or in public settings.

The software environment will include:
\begin{itemize}
  \item A machine learning backend trained on impaired speech datasets, capable of running locally or through cloud-based services.
  \item A user interface layer on the device to display the transcribed text and execute mapped commands.
  \item Integration with existing operating system accessibility frameworks to enable device-level control (e.g., opening apps, sending messages, browsing).
\end{itemize}


\section{Goals}
The table below contains the goals that VoiceBridge must achieve in order to be considered acceptable and complete.

\begin{table}[hp]
\caption{Minimum Viable Product Goals} 
\begin{tabularx}{\textwidth}{lXX} \\[-8pt]
\toprule
\textbf{Goal} & \textbf{Explanation} & \textbf{Reasoning}\\
\midrule
Accurate Transcription & The system must reliably convert impaired or atypical speech audio into text output with measurable accuracy (70\%+ translation accuracy). & Without accurate transcription, the product fails its core purpose of enabling users to communicate.\\
Real-Time Processing & The system should process speech input and generate text/commands with minimal delay (\textless1 second delay). & Enables natural interaction with devices, making the system practical for everyday use.\\
Usability of Input & The product will accept speech input through a standard microphone without requiring specialized equipment. & Keeps the solution accessible and cost-effective, lowering barriers for adoption.\\
Non-Disruptive Feedback & The transcribed text should be displayed clearly to the user for verification. & Transparency builds user trust and allows them to correct errors if necessary.\\
\bottomrule
\end{tabularx}
\end{table}

\section{Stretch Goals}
The table below shows the goals that are not necessary for the product to be viable, but will provide added benefits should they be achieved.

\begin{table}[hp]
\caption{Stretch Goals}
\begin{tabularx}{\textwidth}{lXX} \\[-8pt]
\toprule
\textbf{Goal} & \textbf{Explanation} & \textbf{Reasoning}\\
\midrule
Device Control Integration & Transcribed text should be mapped to commands that can interact with applications on a phone or computer. & Extends the system from communication-only to full device control, empowering users with independence.\\
Personalization & Allow users to customize command mappings (e.g., ``open browser'' or ``send text''). & Personalization increases usability and adapts the system to individual needs and preferences.\\
Robustness to Noise & System should handle moderate background noise in typical environments (home, clinic, public). & Ensures reliability in real-world conditions, not just in quiet test settings.\\
Autonomy & Once activated, the system should allow users to operate devices without requiring assistance from caregivers or family. & Maximizes user independence and supports the core accessibility motivation of the project.\\
\bottomrule
\end{tabularx}
\end{table}

\newpage

\section{Extras}
\begin{itemize}
    \item Literature Review Report
    \item Performance Report
\end{itemize}
The reason why we have a literature review report is to understand the current state of speech recognition technologies, especially those tailored for impaired speech. This report will help us identify existing solutions and pick the most appropriate one we can build on to address the needs of our target stakeholders.
\\
\\
The performance report is essential to evaluate how well our system meets the defined goals. It will provide metrics on accuracy and usability and compare our solution to existing solutions to see whether our solution has made an improvement or not.

\newpage{}

\section*{Appendix --- Reflection}


\begin{enumerate}
    \item What went well while writing this deliverable? \\ \\
    We quickly aligned on the core idea and broke it down into a clear, consolidated problem statement with agreed-upon goals. Using the template kept us focused on abstraction (inputs/outputs, stakeholders, environment) and made it easy to divide.
    \item What pain points did you experience during this deliverable, and how
    did you resolve them? \\ \\
    No major pain-points as this deliverable had a small scope.
    \item How did you and your team adjust the scope of your goals to ensure
    they are suitable for a Capstone project (not overly ambitious but also of
    appropriate complexity for a senior design project)? \\ \\
    We narrowed the scope by defining a concrete MVP and separating stretch features. The MVP focuses on reliable transcription for impaired speech with basic command mapping and clear feedback, while stretch goals (e.g., broader device control, real-time latency targets, personalization, noise robustness) are aspirational, not required for success.
\end{enumerate} 

\section*{References}
%\bibliographystyle{IEEEtran} using references,bib did not work
%\bibliography{../refs/References}
\begin{thebibliography}{00}

\bibitem{nidcd2024assistive}
National Institute on Deafness and Other Communication Disorders,
``Assistive Devices for People with Hearing, Voice, Speech, or Language Disorders,'' 2024.
[Online]. Available: \url{https://www.nidcd.nih.gov/health/assistive-devices-people-hearing-voice-speech-or-language-disorders}.
Accessed: 2024-09-29.

\bibitem{pokel2025variational}
N.~Pokel, P.~Moure, R.~Boehringer, S.-C.~Liu, and Y.~Gao,
``Variational Low-Rank Adaptation for Personalized Impaired Speech Recognition,''
\emph{arXiv preprint} arXiv:2509.20397, 2025.
[Online]. Available: \url{https://arxiv.org/abs/2509.20397}.

\end{thebibliography}
\end{document}