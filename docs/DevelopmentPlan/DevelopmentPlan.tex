\documentclass{article}

\usepackage{booktabs}
\usepackage{tabularx}
\usepackage{comment}
\usepackage{hyperref}


\title{Development Plan\\ Voice Bridge}

\author{\\ Team 13, Speech Buddies}

\date{}

%% Comments

\usepackage{color}

\newif\ifcomments\commentstrue %displays comments
%\newif\ifcomments\commentsfalse %so that comments do not display

\ifcomments
\newcommand{\authornote}[3]{\textcolor{#1}{[#3 ---#2]}}
\newcommand{\todo}[1]{\textcolor{red}{[TODO: #1]}}
\else
\newcommand{\authornote}[3]{}
\newcommand{\todo}[1]{}
\fi

\newcommand{\wss}[1]{\authornote{magenta}{SS}{#1}} 
\newcommand{\plt}[1]{\authornote{cyan}{TPLT}{#1}} %For explanation of the template
\newcommand{\an}[1]{\authornote{cyan}{Author}{#1}}

%% Common Parts

\newcommand{\progname}{Software Engineering} % PUT YOUR PROGRAM NAME HERE
\newcommand{\authname}{Team 13, Speech Buddies
\\ Mazen Youssef
\\ Rawan Mahdi
\\ Luna Aljammal
\\ Kelvin Yu} % AUTHOR NAMES                  

\usepackage{hyperref}
    \hypersetup{colorlinks=true, linkcolor=blue, citecolor=blue, filecolor=blue,
                urlcolor=blue, unicode=false}
    \urlstyle{same}
                                

\setlength{\parindent}{0pt}

\setlength{\parskip}{\baselineskip}

\begin{document}

\maketitle

\begin{table}[hp]
\caption{Revision History} \label{TblRevisionHistory}
\begin{tabularx}{\textwidth}{llX}
\toprule
\textbf{Date} & \textbf{Developer(s)} & \textbf{Change}\\
\midrule
September 22, 2025 & Kelvin Yu & Added Sections 1-5 and updated team charter\\
September 22, 2025 & Rawan Mahdi & Added sections 6 and 9, and parts of charter \\
September 22, 2025 & Luna Aljammal & Added sections 7,8,10,11 and parts of charter \\
... & ... & ...\\
\bottomrule
\end{tabularx}
\end{table}

\newpage{}

\begin{comment}\wss{Put your introductory blurb here.  Often the blurb is a brief roadmap of
what is contained in the report.}
\end{comment}

This report outlines VoiceBridge’s project details, including an overview of the licensing, team norms, proof of concept plan, and the expected technology that’ll be utilized throughout the development phase. 
The github is also linked \href{https://github.com/speech-buddies/VoiceBridge}{here.} 

\begin{comment}\wss{Additional information on the development plan can be found in the
\href{https://gitlab.cas.mcmaster.ca/courses/capstone/-/blob/main/Lectures/L02b_POCAndDevPlan/POCAndDevPlan.pdf?ref_type=heads}
{lecture slides}.}
\end{comment}



\section{Confidential Information}

This project does \textbf{not} currently contain any confidential information from industry. All datasets, tools, and resources used are publicly available and open source.

In the future, if we collect voice data from external participants (e.g., individuals with speech disabilities) and they request that their information remain confidential, we will put a confidentiality and data use agreement / consent form in place before collecting any data. This agreement will outline how their data will be stored, de-identified, and used solely within the project.


\section{IP to Protect}



There is no IP to protect.

\section{Copyright License}

Our team is adopting the MIT License for this project. The license file is included in our repository 
(\href{https://github.com/speech-buddies/VoiceBridge/blob/main/LICENSE}{\texttt{LICENSE}}).

\section{Team Meeting Plan}

Our team will meet twice a week:

\begin{itemize}
    \item In-person meetings on Tuesdays at 12:30 PM
    \item Virtual meetings on Thursdays at 4:30 PM
\end{itemize}

We have also set up weekly meetings with our industry advisor (Dr.~Christian Brodbeck) which will be conducted online every Thursday at 11:00 AM.

In each meeting, every member will share their findings and what they worked on since the previous meeting. A designated chair will lead the discussion following a prepared agenda, and meeting notes will be recorded to track decisions, progress, and action items.

\section{Team Communication Plan}

Our team will use multiple channels to stay organized and collaborate effectively:

\begin{itemize}
    \item \textbf{Discord} – for quick day-to-day communication and discussions
    \item \textbf{Notion (shared space)} – for organizing documents and resources, taking meeting notes, and task planning
    \item \textbf{GitHub Issues} – for tracking technical tasks, bugs, and progress on the project codebase
\end{itemize}

These tools will ensure that all updates, discussions, and decisions are documented and accessible to all team members.

\section{Team Member Roles}

% \wss{You should identify the types of roles you anticipate, like notetaker,
% leader, meeting chair, reviewer.  Assigning specific people to those roles is
% not necessary at this stage.  In a student team the role of the individuals will
% likely change throughout the year.}

All team members will be responsible for creating issues, developing, testing, documenting, and reviewing code. Development work will be partitioned based on project feature. Administrative tasks will be broken down by the following roles, which will be cycled on a regular basis.

\subsection{Project Lead}

\begin{itemize}
  \item Chair meetings with team and supervisors by preparing agenda and guiding discussions
  \item Liaison with supervisors
  \item Ensures team is progressing towards project goals each week
\end{itemize}

\subsection{Content Manager}

\begin{itemize}
  \item Collects materials needed for proposals for data usage, ethics agreements, etc
  \item Documents papers, datasets, and all other resources referenced by the team throughout the project
  \item Updates Kanban board each week
\end{itemize}

\subsection{Book Keeper}

\begin{itemize}
  \item Takes notes during team meetings and supervisor meetings
  \item Highlights upcoming deliverables and deadlines
  \item Submits deliverables if needed
\end{itemize}

\subsection{Hardware Resource Manager }

\begin{itemize}
  \item Tracks and documents expenses related to hardware usage
\end{itemize}


\section{Workflow Plan}

  \begin{comment}
	\item How will you be using git, including branches, pull request, etc.?
	\item How will you be managing issues, including template issues, issue
	classification, etc.?
  \item Use of CI/CD
  \item 
\end{comment}
  
The team will use Git and GitHub for version control, to manage issues, and to handle pull requests. During the weekly meetings, the team will create a list of issues to complete. This will follow an issue template with a description that includes the Definition of Done (DoD), an assignee, a reviewer, as well as the source/purpose of the issue and any related dependencies or issues. Issues will follow a naming convention which will be included in the associated development branch. Commits will also include the issue number for traceability purposes.

\noindent To keep an organized project, PRs will be merged to main after a reviewer approves that the DoD has been met. Tests will be required where applicable in the DoD. The issue will be linked in the PR description.

Issues will be used to track all tasks, such as:
\begin{itemize}
    \item Development
    \item Tests
    \item Team meetings
    \item Deliverable Submissions
    \item Infrastructure Setup
    \item Resolving Bugs
    \item Addressing Supervisor Feedback
\end{itemize}

The branch names will include a prefix to specify the type of issue worked on, including:
\begin{itemize}
    \item feature/*
    \item bug/*
    \item infra/*
    \item docs/*
\end{itemize}

Once the assignee is confident with their completed issue, they will notify the reviewer and address potential feedback/concerns the reviewer may raise. After the assignee and reviewer agree on the issue completion, the issue owner will notify the team that it is complete.


Our team will use GitHub Actions to automate testing and maintain code quality standards. The CI pipeline will run unit tests on core functionality to ensure that bugs aren’t introduced, and existing components are working as expected. A linting tool will be added to maintain consistent styling and to review code for potential errors.

PRs must pass all tests and lint checks before developers request review. Branch protection rules will prevent merging until a minimum of 1 PR approval is added.

\section{Project Decomposition and Scheduling}
\begin{comment}
\begin{itemize}
  \item How will you be using GitHub projects?
  \item Include a link to your GitHub project
\end{itemize}

\wss{How will the project be scheduled?  This is the big picture schedule, not
details. You will need to reproduce information that is in the course outline
for deadlines.}
\end{comment}

The project will be organized on GitHub projects, with a linked Kanban-board to visualize issue completion. Swimlanes will be used to show the stage of the issue, including:

\begin{itemize}
    \item \textbf{Backlog:} issues which have been identified and created
    \item \textbf{To Do:} lists issues which need to be done before the next meeting, have been groomed by 1 or more team members
    \item \textbf{In Progress:} issues that have been started
    \item \textbf{In Review:} development is complete, awaiting feedback (assigned to reviewer)
    \item \textbf{Done:} DoD is met, and team is informed of completion
\end{itemize}

Link to GitHub project: \url{https://github.com/speech-buddies/VoiceBridge}

\section{Proof of Concept Demonstration Plan}

% What is the main risk, or risks, for the success of your project?  What will you
% demonstrate during your proof of concept demonstration to convince yourself that
% you will be able to overcome this risk?

The main risk is the difficulty of tuning a machine learning model to accurately transform dysarthric speech into text. The success of our project requires adapting an existing model to transcribe the atypical and inconsistent speech patterns. To overcome this risk, we will orient our proof of concept around testing the transcription accuracy of various architectural components. The prototype will be successful if it produces text that translates the speech content with 70\% accuracy. This will provide us with the assurance that our tuning and feature extraction methods can improve atypical speech recognition.

We will attempt to generate this model using the following techniques, and compare and contrast the results to build the final architecture of our model:
\begin{enumerate}
    \item Fine-tune selected layers (input/output) of an open weight Speech-to-Text (STT) model, such as Whisper ASR, DeepSpeech, or Wav2Vec using the open Torgo dataset
    \item Perform preprocessing steps on the speech wave forms to amplify features in the dysarthric speech that improve ASR accuracy 
\end{enumerate}


We plan to continue to research the problem space, and with the help of our supervisor, Dr.Brodbeck, identify more techniques and architectural components that can help us reach our goal.

\section{Expected Technology}
\begin{comment}
\wss{What programming language or languages do you expect to use?  What external
libraries?  What frameworks?  What technologies.  Are there major components of
the implementation that you expect you will implement, despite the existence of
libraries that provide the required functionality.  For projects with machine
learning, will you use pre-trained models, or be training your own model?  }

\wss{The implementation decisions can, and likely will, change over the course
of the project.  The initial documentation should be written in an abstract way;
it should be agnostic of the implementation choices, unless the implementation
choices are project constraints.  However, recording our initial thoughts on
implementation helps understand the challenge level and feasibility of a
project.  It may also help with early identification of areas where project
members will need to augment their training.}

Topics to discuss include the following:

\begin{itemize}
\item Specific programming language
\item Specific libraries
\item Pre-trained models
\item Specific linter tool (if appropriate)
\item Specific unit testing framework
\item Investigation of code coverage measuring tools
\item Specific plans for Continuous Integration (CI), or an explanation that CI
  is not being done
\item Specific performance measuring tools (like Valgrind), if
  appropriate
\item Tools you will likely be using?
\end{itemize}

\wss{git, GitHub and GitHub projects should be part of your technology.}
\end{comment}

We expect to use \textbf{Python} as the programming language. 

We plan to use \textbf{PyTorch} or \textbf{TensorFlow} for model fine-tuning. We will extend and fine-tune existing models trained on standard speech, adapting them to accurately process dysarthric speech.

We will use the \textbf{PyLint} linter to ensure pull requests meet coding standards.

Testing will be done with \textbf{Pytest}.

For code coverage, we will use \textbf{coverage.py}.

\bigskip
\noindent\textbf{Environment:} For model training, proof of concepts, and performance testing, we will use Google Colab for convenience and ease of use. The final application will likely require a managed Python environment, such as \textbf{Conda}, to handle custom dependencies and allow easy local integration of project components.

\bigskip
\noindent\textbf{Compute Resources:} Since the project may involve handling large amounts of data during training and tuning, we plan to use McMaster CAS GPU clusters. Alternatively, we may consider Google Colab Pro, which offers pay-as-you-go compute units.


\section{Coding Standard}

%\wss{What coding standard will you adopt?}
We will follow the \href{https://peps.python.org/pep-0008/}{PEP 8} coding standard.


\newpage{}

\section*{Appendix --- Reflection}


The purpose of reflection questions is to give you a chance to assess your own
learning and that of your group as a whole, and to find ways to improve in the
future. Reflection is an important part of the learning process.  Reflection is
also an essential component of a successful software development process.  

Reflections are most interesting and useful when they're honest, even if the
stories they tell are imperfect. You will be marked based on your depth of
thought and analysis, and not based on the content of the reflections
themselves. Thus, for full marks we encourage you to answer openly and honestly
and to avoid simply writing ``what you think the evaluator wants to hear.''

Please answer the following questions.  Some questions can be answered on the
team level, but where appropriate, each team member should write their own
response:


\begin{enumerate}
    \item Why is it important to create a development plan prior to starting the
    project?

    It encourages the team to establish collaboration practices, align expectations, and identify potential risks early. A development plan sets clear guidelines for communication, workflows, coding standards, and scheduling, which helps reduce misunderstandings later in the project. It also gives the team a shared reference document, ensuring that every member knows their responsibilities, how decisions will be made, and what tools will be used. By agreeing on these practices upfront, the team saves time, avoids unnecessary conflict, and can focus on solving the technical challenges of the project.

    \item In your opinion, what are the advantages and disadvantages of using
    CI/CD?
    
    Luna:

CI/CD simplifies the integration of changes by automating the repetitive tasks of building and testing code, reducing the manual effort needed. It provides insights on the project’s health, making it easier to track and resolve conflicts early. 

However, using CI/CD is ineffective without maintaining an up-to-date testing suite. Configuring the pipeline and resolving issues could be time consuming and might slow development progress.

Rawan: The use of CI/CD enforces development standards from the very start of its implementation, if used correctly. It needs to be scaled with the project (e.g. creating a workflow to push artifacts to cloud once a new version is ready) to prevent too much development overhead. 

Kelvin: 

CI/CD makes development smoother by catching problems early through automated builds and tests. It helps keep the codebase stable, reduces integration issues, and allows new features to be delivered more quickly. However, setting up pipelines can take a lot of time and effort, and they require ongoing maintenance as the project evolves. Tests may be flaky or slow, which can block progress and frustrate developers. Pipelines also consume computing resources and can become slower as the project grows.

Mazen:

CI/CD helps you ship with confidence because builds, tests, and deploys run the same way every time. You get fast feedback, smaller changes are easier to review, and you can roll back if something slips through. With the right gates in place tests, linting, and basic security checks, the codebase stays healthier and environments stay consistent.

The catch is that CI/CD only works well if the team commits the additional effort to creating good tests and keeping the pipeline tidy. Pipelines can get slow or flaky, secrets need careful handling, costs can creep up as jobs and artifacts grow, and it’s easy to lean on a green check instead of doing thoughtful reviews and exploratory testing.

    \item What disagreements did your group have in this deliverable, if any,
    and how did you resolve them?

    We did not have disagreements in this deliverable. We were able to delegate tasks, and complete them ahead of our meetings, which allowed us to review the work when we met. This helped us stay on track, and avoid any conflicts. 
\end{enumerate}

\newpage{}

\section*{Appendix --- Team Charter}



\subsection*{External Goals}

\begin{comment}
\wss{What are your team's external goals for this project? These are not the
goals related to the functionality or quality fo the project.  These are the
goals on what the team wishes to achieve with the project.  Potential goals are
to win a prize at the Capstone EXPO, or to have something to talk about in
interviews, or to get an A+, etc.}
\end{comment}

\subsection*{Attendance}

\subsubsection*{Expectations}

\begin{comment}
\wss{What are your team's expectations regarding meeting attendance (being on
time, leaving early, missing meetings, etc.)?}
\end{comment}
The team expects members to attend all meetings, and to arrive on time. If someone is unable to attend a meeting or may be late, they are expected to inform the team ahead of time, and get review the meeting minutes. They should also provide a status update on their assigned tasks. Prior to each meeting, the project lead will share an agenda to outline the meeting goals. If we cover all the needed topics during a meeting, we can end early. If we need to continue later, we can either continue in the discord chat or schedule a follow-up meeting

\subsubsection*{Acceptable Excuse}

\begin{comment}
\wss{What constitutes an acceptable excuse for missing a meeting or a deadline?
What types of excuses will not be considered acceptable?}
\end{comment}
An acceptable excuse would be an unexpected emergency that the individual could not have predicted. Otherwise, if someone anticipates that they cannot join a meeting, they should communicate with the team ahead of time, and consider rescheduling. It is unacceptable to miss a meeting due to poor time-management.

\subsubsection*{In Case of Emergency}

\begin{comment}
\wss{What process will team members follow if they have an emergency and cannot
attend a team meeting or complete their individual work promised for a team
deliverable?}
\end{comment}
They will inform the team as soon as possible of this situation. For the first occurence, the team will delegate their responsibilities amongst each other. However, this should not occur frequently, as it could slow down team progress, and impact performance. If the individual struggles to meet several deadlines, the individual should discuss their situation with the instructors.

\subsection*{Accountability and Teamwork}

\subsubsection*{Quality} 

\begin{comment}
\wss{What are your team's expectations regarding the quality
of team members' preparation for team meetings and the quality of the
deliverables that members bring to the team?}
\end{comment}

The quality of work should adhere to level 4 of rubrics. It should be completed to the best of the team's abilities. If the team is inexperienced in a domain, they should use online resources to learn about it and complete high quality work. Tasks should be completed ahead of meetings, to allow for feedback and discussions during meetings.
\subsubsection*{Attitude}
Team members are expected to maintain a respectful and supportive attitude. Feedback should be constructive, and disagreements should be handled professionally.

\begin{comment}
\wss{What are your team's expectations regarding team members' ideas,
interactions with the team, cooperation, attitudes, and anything else regarding
team member contributions?  Do you want to introduce a code of conduct?  Do you
want a conflict resolution plan?  Can adopt existing codes of conduct.}
\end{comment}

\subsubsection*{Stay on Track}
Each member is responsible for completing their assigned tasks by the agreed deadlines. If someone anticipates delays, they must communicate early so adjustments can be made. Meetings will be used to review progress and ensure alignment with project goals, helping the team avoid falling behind.

\begin{comment}
\wss{What methods will be used to keep the team on track? How will your team
ensure that members contribute as expected to the team and that the team
performs as expected? How will your team reward members who do well and manage
members whose performance is below expectations?  What are the consequences for
someone not contributing their fair share?}
\end{comment}


\begin{comment}
\wss{You may wish to use the project management metrics collected for the TA and
instructor for this.}
\end{comment}


\begin{comment}
\wss{You can set target metrics for attendance, commits, etc.  What are the
consequences if someone doesn't hit their targets?  Do they need to bring the
coffee to the next team meeting?  Does the team need to make an appointment with
their TA, or the instructor?  Are there incentives for reaching targets early?}
\end{comment}

\subsubsection*{Team Building}
The team will actively work to build a collaborative environment through open communication, mutual support, and recognition of contributions.

\begin{comment}
\wss{How will you build team cohesion (fun time, group rituals, etc.)? }
\end{comment}
Our team will meet once a week in person to facilitate building relationships. We also plan to schedule a bi-monthly activity. 
\subsubsection*{Decision Making} 

\begin{comment}
\wss{How will you make decisions in your group? Consensus?  Vote? How will you
handle disagreements? }

\end{comment}
Decisions will be made as a team by working towards a consensus, after sufficient research and consultation with supervisors. If a consensus cannot be reached, the team will take a vote on the decision.

\end{document}
