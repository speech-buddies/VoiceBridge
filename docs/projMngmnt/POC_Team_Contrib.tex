\documentclass{article}

\usepackage{float}
\restylefloat{table}

\usepackage{booktabs}

\title{Team Contributions: POC\\\progname}

\author{\authname}

\date{}

\input{../Comments}
%% Common Parts

\newcommand{\progname}{Software Engineering} % PUT YOUR PROGRAM NAME HERE
\newcommand{\authname}{Team 13, Speech Buddies
\\ Mazen Youssef
\\ Rawan Mahdi
\\ Luna Aljammal
\\ Kelvin Yu} % AUTHOR NAMES                  

\usepackage{hyperref}
    \hypersetup{colorlinks=true, linkcolor=blue, citecolor=blue, filecolor=blue,
                urlcolor=blue, unicode=false}
    \urlstyle{same}
                                


\begin{document}

\maketitle

This document summarizes the contributions of each team member up to the POC
Demo.  The time period of interest is the time between the beginning of the term
and the POC demo.

\section{Demo Plans}

% \wss{What will you be demonstrating}
For the POC Demo, we will be demonstrating the following results of our analysis and experimentation, as described in the \href{https://github.com/speech-buddies/VoiceBridge/blob/main/docs/DevelopmentPlan/DevelopmentPlan.pdf}{Development Plan document}:

\begin{enumerate}
    \item The word error rate (WER) on the highest accuracy open source ASR models (Whisper/Whisper-small, Wav2Vec, etc.)
    \item An improved accuracy (decreased WER) proportional to some \% of training -  considering the machine learning scaling laws, this will give us an accurate approximation of what the model accuracy will be after training on the full dataset.
    \item A demonstration of how preprocessing the dysarthric speech data may improve the model's transcription accuracy (may be demoed on tuned or un-tuned models)
    \item A full training workflow plan, describing the resources and techniques we will adopt to handle the large amount of training parameters, training data, and need for high performance compute.
\end{enumerate}

\section{Team Meeting Attendance}

% \wss{For each team member how many team meetings have they attended over the
% time period of interest.  This number should be determined from the meeting
% issues in the team's repo.  The first entry in the table should be the total
% number of team meetings held by the team.}


\begin{table}[H]
\centering
\begin{tabular}{ll}
\toprule
\textbf{Student} & \textbf{Meetings}\\
\midrule
Total & 11\\
Kelvin Yu & 10\\
Luna Aljammal & 10\\
Mazen Youssef & 11\\
Rawan Mahdi & 11\\
\bottomrule
\end{tabular}
\end{table}

% \wss{If needed, an explanation for the counts can be provided here.}

\section{Supervisor/Stakeholder Meeting Attendance}

% \wss{For each team member how many supervisor/stakeholder team meetings have
% they attended over the time period of interest.  This number should be determined
% from the supervisor meeting issues in the team's repo.  The first entry in the
% table should be the total number of supervisor and team meetings held by the
% team.  If there is no supervisor, there will usually be meetings with
% stakeholders (potential users) that can serve a similar purpose.}

% \noindent \textbf{Supervisor's Name: } Dr. Christian Brodbeck

\begin{table}[H]
\centering
\begin{tabular}{ll}
\toprule
\textbf{Student} & \textbf{Meetings}\\
\midrule
Total & 6\\
Kelvin Yu & 6\\
Luna Aljammal & 6\\
Mazen Youssef & 6\\
Rawan Mahdi & 6\\
\bottomrule
\end{tabular}
\end{table}

% \wss{If needed, an explanation for the counts can be provided here.}

\section{Lecture Attendance}

% \wss{For each team member how many lectures have they attended over the time
% period of interest.  This number should be determined from the lecture issues in
% the team's repo. You can find the number of lectures in the time period of
% interest by looking at the
% \href{https://calendar.google.com/calendar/u/0/embed?src=rnboqiaki1k2la7rpu3bn0um58@group.calendar.google.com&ctz=America/Toronto}
% {Google calendar} for the capstone course.}

% \wss{NOTE: There will be approximately 13 lectures between the start of class
% and the POC demos}

\begin{table}[H]
\centering
\begin{tabular}{ll}
\toprule
\textbf{Student} & \textbf{Lectures}\\
\midrule
Total & 10\\
Kelvin Yu & 5\\
Luna Aljammal & 5\\
Mazen Youssef & 5\\
Rawan Mahdi & 5\\
\bottomrule
\end{tabular}
\end{table}

Our team coordinated lecture attendance by alternating members to ensure that at least two members attended each session and took notes to share with the rest of the group.  
This approach allowed us to stay consistent despite midterms and other scheduling constraints.  
We missed three lectures in total due to overlapping supervisor meetings, which were the only times that worked for our supervisor.  
For these sessions, we reviewed the lecture slides and materials afterward to remain aligned with course content.  
Outside of these instances, attendance and lecture content were regularly communicated through our group chat and weekly meetings, ensuring all members remained informed and up to date.
%\wss{If needed, an explanation for the lecture attendance can be provided here.}

\section{TA Document Discussion Attendance}


\noindent \textbf{TA's Name: } Tanya Djavaherpour

\begin{table}[H]
\centering
\begin{tabular}{ll}
\toprule
\textbf{Student} & \textbf{Lectures}\\
\midrule
Total & 3\\
Mazen Youssef & 3\\
Rawan Mahdi & 3\\
Luna Aljammal & 3\\
Kelvin Yu & 3\\
\bottomrule
\end{tabular}
\end{table}


\section{Commits}

\begin{table}[H]
\centering
\begin{tabular}{lll}
\toprule
\textbf{Student} & \textbf{Commits} & \textbf{Percent}\\
\midrule
Total & 109 & 100\% \\
Mazen Youssef & 17 & 16\% \\
Rawan Mahdi & 31 & 28\% \\
Luna Aljammal & 44 & 40\% \\
Kelvin Yu & 17 & 16\% \\
\bottomrule
\end{tabular}
\end{table}

At the start of capstone, Luna and Rawan made more frequent, smaller commits, while Mazen and Kelvin made fewer but larger commits. 

\section{Issue Tracker}

\begin{table}[H]
\centering
\begin{tabular}{lll}
\toprule
\textbf{Student} & \textbf{Authored (O+C)} & \textbf{Assigned (C only)}\\
\midrule
Mazen Youssef& 9 & 13 \\
Rawan Mahdi & 9 & 17 \\
Luna Aljammal & 16 & 12 \\
Kelvin Yu & 6 & 7   \\
\bottomrule
\end{tabular}
\end{table}


\section{CICD}

\wss{Say how CICD will be used in your project}

\section{Team Charter Trigger Items}

\wss{Provide a summary of the quantified triggers identified in the team's
charter.}

\wss{Provide a list of any violations of the triggers.  If the team wishes, the
violations can be summarized on aggregate, instead of naming specific team
members.}

\wss{Provide a plan to address the violations.  This could include revising the
triggers, if they are found to be too weak, strong or ambiguous.}

\section{Additional Productivity Metrics}

\wss{If your team has additional metrics of productivity, please feel free to
add them to this report.}

\end{document}