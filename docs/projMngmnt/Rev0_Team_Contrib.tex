\documentclass{article}

\usepackage{float}
\restylefloat{table}

\usepackage{booktabs}

\title{Team Contributions: Rev 0\\\progname}

\author{\authname}

\date{}

\input{../Comments}
%% Common Parts

\newcommand{\progname}{Software Engineering} % PUT YOUR PROGRAM NAME HERE
\newcommand{\authname}{Team 13, Speech Buddies
\\ Mazen Youssef
\\ Rawan Mahdi
\\ Luna Aljammal
\\ Kelvin Yu} % AUTHOR NAMES                  

\usepackage{hyperref}
    \hypersetup{colorlinks=true, linkcolor=blue, citecolor=blue, filecolor=blue,
                urlcolor=blue, unicode=false}
    \urlstyle{same}
                                


\begin{document}

\maketitle

This document summarizes the contributions of each team member for the Rev 0
Demo.  The time period of interest is the time between the PoC demo and the Rev
0 demo; the contributions prior to the PoC are NOT included.

\section{Demo Plans}

\wss{What will you be demonstrating}

\section{Team Meeting Attendance}

\wss{For each team member how many team meetings have they attended over the
time period of interest.  This number should be determined from the meeting
issues in the team's repo.  The first entry in the table should be the total
number of team meetings held by the team.}

\begin{table}[H]
\centering
\begin{tabular}{ll}
\toprule
\textbf{Student} & \textbf{Meetings}\\
\midrule
Total & Num\\
Name 1 & Num\\
Name 2 & Num\\
Name 3 & Num\\
Name 4 & Num\\
Name 5 & Num\\
\bottomrule
\end{tabular}
\end{table}

\wss{If needed, an explanation for the counts can be provided here.}

\section{Supervisor/Stakeholder Meeting Attendance}

\wss{For each team member how many supervisor/stakeholder team meetings have
they attended over the time period of interest.  This number should be determined
from the supervisor meeting issues in the team's repo.  The first entry in the
table should be the total number of supervisor and team meetings held by the
team.  If there is no supervisor, there will usually be meetings with
stakeholders (potential users) that can serve a similar purpose.}

\noindent \textbf{Supervisor's Name: } [fill in this information]

\begin{table}[H]
\centering
\begin{tabular}{ll}
\toprule
\textbf{Student} & \textbf{Meetings}\\
\midrule
Total & Num\\
Name 1 & Num\\
Name 2 & Num\\
Name 3 & Num\\
Name 4 & Num\\
Name 5 & Num\\
\bottomrule
\end{tabular}
\end{table}

\wss{If needed, an explanation for the counts can be provided here.}

\section{Lecture Attendance}

\wss{For each team member how many lectures have they attended over the time
period of interest.  This number should be determined from the lecture issues in
the team's repo. You can find the number of lectures in the time period of
interest by looking at the
\href{https://calendar.google.com/calendar/u/0/embed?src=rnboqiaki1k2la7rpu3bn0um58@group.calendar.google.com&ctz=America/Toronto}
{Google calendar} for the capstone course.}

\wss{NOTE: There will be approximately 1 lecture between the POC and Rev0 demos}

\begin{table}[H]
\centering
\begin{tabular}{ll}
\toprule
\textbf{Student} & \textbf{Lectures}\\
\midrule
Total & Num\\
Name 1 & Num\\
Name 2 & Num\\
Name 3 & Num\\
Name 4 & Num\\
Name 5 & Num\\
\bottomrule
\end{tabular}
\end{table}

\wss{If needed, an explanation for the lecture attendance can be provided here.}

\section{TA Document Discussion Attendance}

\wss{For each team member how many of the informal document discussion meetings
with the TA were attended over the time period of interest.}

\noindent \textbf{TA's Name: } [fill in this information]

\begin{table}[H]
\centering
\begin{tabular}{ll}
\toprule
\textbf{Student} & \textbf{Lectures}\\
\midrule
Total & Num\\
Name 1 & Num\\
Name 2 & Num\\
Name 3 & Num\\
Name 4 & Num\\
Name 5 & Num\\
\bottomrule
\end{tabular}
\end{table}

\wss{If needed, an explanation for the attendance can be provided here.}

\section{Commits}

\wss{For each team member how many commits to the main branch have been made
over the time period of interest.  The total is the total number of commits for
the entire team since the beginning of the term.  The percentage is the
percentage of the total commits made by each team member.}

\begin{table}[H]
\centering
\begin{tabular}{lll}
\toprule
\textbf{Student} & \textbf{Commits} & \textbf{Percent}\\
\midrule
Total & Num & 100\% \\
Name 1 & Num & \% \\
Name 2 & Num & \% \\
Name 3 & Num & \% \\
Name 4 & Num & \% \\
Name 5 & Num & \% \\
\bottomrule
\end{tabular}
\end{table}

\wss{If needed, an explanation for the counts can be provided here.  For
instance, if a team member has more commits to unmerged branches, these numbers
can be provided here.  If multiple people contribute to a commit, git allows for
multi-author commits.}

\section{Issue Tracker}

\begin{table}[H]
\centering
\begin{tabular}{lll}
\toprule
\textbf{Student} & \textbf{Authored (O+C)} & \textbf{Assigned (C only)}\\
\midrule
Luna & 14 & 19 \\
Kelvin & 7 & 14 \\
Mazen & 11 & 17 \\
Rawan & 9  & 26 \\
\bottomrule
\end{tabular}
\end{table}

The difference in the Authored column is attributed to the creation of team meeting issues, which contributes significantly to the counts for some members. 



\section{CICD}

Two GitHub Actions workflows run on the \texttt{main} branch for both push and pull events:
\begin{itemize}
    \item \textbf{buildtex}: Triggered when files under \texttt{docs/**} change. This workflow builds the modified \LaTeX{} files into PDF artifacts and commits the generated PDFs back to the \texttt{main} branch.
    \item \textbf{Python CI}: Triggered when \texttt{.py} files, \texttt{requirements.txt}, or the workflow configuration change. This workflow uses Python~3.11 and runs \texttt{pylint} and \texttt{pytest} to validate code quality and correctness.
\end{itemize}


\section{Team Charter Trigger Items}

The following quantified triggers are established in our team charter:
\begin{itemize}
    \item \textbf{Meeting Attendance:} All members are expected to attend every meeting unless they have communicated in advance.
    \item \textbf{Timely Task Updates:} Assigned tasks should be completed on time or any delays must be communicated ahead through Discord.
    \item \textbf{Review Before Merge:} No code or document changes may be merged without review and approval from at least one other team member.
\end{itemize}

There have been no violations of the team charter triggers during this reporting period. As no violations have occurred, no corrective actions are currently necessary. 


\section{Additional Productivity Metrics}
N/A as of this moment.

\end{document}