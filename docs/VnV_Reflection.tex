\begin{enumerate}
  \item \textbf{What went well while writing this deliverable?} \\
  The writing process went smoothly because there was strong alignment between the project requirements and the testing approach.
  Breaking down each functional requirement into specific, well-defined tests helped focus the effort and ensured completeness. The collaboration
  within the team was productive, allowing us to clarify assumptions and verify tool use early on. Being able to connect test cases directly to the
  requirements and think through measurable success criteria gave the plan a solid foundation.

  \item \textbf{What pain points did you experience during this deliverable, and how did you resolve them?} \\
  A major difficulty was ensuring that all planned tests were realistically feasible given our current toolset, resources, and project timeline.
  Some initial test ideas were technically interesting but would have required tools or integrations we didn’t yet have, or would have taken more time than we 
  could allocate. To address this, we carefully reviewed past project documentation and assessed our CI/CD and automation capabilities to see what was
  immediately implementable. We then adjusted the scope of certain tests, streamlining some, combining others, and prioritizing high-impact cases, so that every
  test included in the plan was both practical and meaningful. This approach allowed us to maintain ambitious coverage without overcommitting, ensuring that the V\&V plan is
 actionable, realistic, and aligned with project constraints.

  \item \textbf{What knowledge and skills will the team collectively need to acquire to successfully complete the verification and validation of your project?} \\
  To successfully complete the verification and validation of the VoiceBridge project, the team will need to acquire a combination of technical, ethical, and user-centered skills. 
  This includes understanding speech data quality standards to verify the clarity, consistency, and accuracy of dysarthric speech recordings, along with proficiency in labeling and
   annotation tools to ensure reliable datasets. The team must also develop skills in machine learning verification techniques, such as cross-validation, confusion matrix analysis, and bias detection, to evaluate
   model performance objectively. Knowledge of accessibility and ethical guidelines will be essential to ensure responsible data collection, participant consent, and inclusivity for users with speech impairments.
   In addition, team members should strengthen their understanding of software testing practices, including unit, integration, and system testing, to verify code reliability.
   Finally, the ability to conduct usability testing and analyze user feedback will be crucial for validating the system’s real-world effectiveness and ensuring that VoiceBridge
   genuinely improves accessibility and communication support for its intended users.
\end{enumerate}
