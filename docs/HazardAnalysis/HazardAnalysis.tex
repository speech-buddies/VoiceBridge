\documentclass{article}

\usepackage{booktabs}
\usepackage{tabularx}
\usepackage{hyperref}

\hypersetup{
    colorlinks=true,       % false: boxed links; true: colored links
    linkcolor=red,          % color of internal links (change box color with linkbordercolor)
    citecolor=green,        % color of links to bibliography
    filecolor=magenta,      % color of file links
    urlcolor=cyan           % color of external links
}

\title{Hazard Analysis\\\progname}

\author{\authname}

\date{}

%% Comments

\usepackage{color}

\newif\ifcomments\commentstrue %displays comments
%\newif\ifcomments\commentsfalse %so that comments do not display

\ifcomments
\newcommand{\authornote}[3]{\textcolor{#1}{[#3 ---#2]}}
\newcommand{\todo}[1]{\textcolor{red}{[TODO: #1]}}
\else
\newcommand{\authornote}[3]{}
\newcommand{\todo}[1]{}
\fi

\newcommand{\wss}[1]{\authornote{magenta}{SS}{#1}} 
\newcommand{\plt}[1]{\authornote{cyan}{TPLT}{#1}} %For explanation of the template
\newcommand{\an}[1]{\authornote{cyan}{Author}{#1}}

%% Common Parts

\newcommand{\progname}{Software Engineering} % PUT YOUR PROGRAM NAME HERE
\newcommand{\authname}{Team 13, Speech Buddies
\\ Mazen Youssef
\\ Rawan Mahdi
\\ Luna Aljammal
\\ Kelvin Yu} % AUTHOR NAMES                  

\usepackage{hyperref}
    \hypersetup{colorlinks=true, linkcolor=blue, citecolor=blue, filecolor=blue,
                urlcolor=blue, unicode=false}
    \urlstyle{same}
                                


\begin{document}

\maketitle
\thispagestyle{empty}

~\newpage

\pagenumbering{roman}

\begin{table}[hp]
\caption{Revision History} \label{TblRevisionHistory}
\begin{tabularx}{\textwidth}{llX}
\toprule
\textbf{Date} & \textbf{Developer(s)} & \textbf{Change}\\
\midrule
October 7, 2025 & Kelvin Yu & Added Sections 1-4\\
Date2 & Name(s) & Description of changes\\
... & ... & ...\\
\bottomrule
\end{tabularx}
\end{table}

~\newpage

\tableofcontents

~\newpage

\pagenumbering{arabic}


\section{Introduction}

In the context of VoiceBridge, a hazard is any condition or event that could lead to harm, system failure, or loss of functionality during the collection, processing, and communication of speech data. 
The purpose of this analysis is to identify, evaluate, and propose mitigations for potential hazards before and during the system’s lifecycle.


\section{Scope and Purpose of Hazard Analysis}

Our hazard analysis identifies potential risks that could compromise the VoiceBridge system throughout its development lifecycle and operational use.

\vspace{3mm}
\noindent Potential losses that could result from identified hazards include:
\begin{itemize}
    \item Loss of user trust due to poor accuracy, delayed feedback, or confusing interactions. 
    \item Loss of confidentiality if sensitive voice data, transcripts, or credentials are leaked.
    \item Loss of availability caused by hardware, software, or network failures that make the system unusable.
    \item Loss of integrity if documents or transcripts are modified, duplicated, or misinterpreted.
\end{itemize}

\noindent This scope ensures that both technical and human-interaction factors are considered when evaluating safety and usability risks.

\section{System Boundaries and Components}

VoiceBridge is a multi-component software system designed to convert and interpret speech reliably while protecting user data. 

\vspace{3mm}
\noindent The main components analyzed for hazards are:

\begin{itemize}
    \item Input Devices: Microphones and recording interfaces used to capture audio.
    \item ML Model Pipeline: The models responsible for phoneme recognition, intent detection, and command processing.
    \item Database/Storage: Secure repositories for user data, transcriptions, and system logs.
    \item Communication Layer: Real-time streaming, API connections, and message handling.
    \item User Interface: The front-end application where users interact and receive feedback.
    \item Third-Party Libraries and Services: External dependencies for authentication, hosting, and model frameworks.
\end{itemize}

\noindent Each of these components forms a boundary for potential hazards and has been evaluated for risks such as data loss, model inaccuracy, or accessibility failures.

\section{Critical Assumptions}

The following assumptions define the environment in which VoiceBridge operates and help focus this analysis on realistic and manageable risks:

\begin{itemize}
    \item User devices and microphones are assumed to function normally, though temporary access loss or permission issues may occur and must be handled gracefully.
    
    \item Internet and cloud services are assumed to be available most of the time, but brief outages may happen. VoiceBridge must recover automatically.
    
    \item Users may provide imperfect audio (e.g., background noise or unclear pronunciation), and the system must tolerate and adapt to such inputs.
    
    \item Data storage systems and APIs are protected by authentication and encryption, reducing but not removing data-breach risks.
    
    \item Third-party and open-source libraries are assumed to be maintained and free from critical vulnerabilities, though version control and verification are required.
\end{itemize}


\section{Failure Mode and Effect Analysis}

\wss{Include your FMEA table here. This is the most important part of this document.}
\wss{The safety requirements in the table do not have to have the prefix SR.
The most important thing is to show traceability to your SRS. You might trace to
requirements you have already written, or you might need to add new
requirements.}
\wss{If no safety requirement can be devised, other mitigation strategies can be
entered in the table, including strategies involving providing additional
documentation, and/or test cases.}

\section{Safety and Security Requirements}

\wss{Newly discovered requirements.  These should also be added to the SRS.  (A
rationale design process how and why to fake it.)}

\section{Roadmap}

\wss{Which safety requirements will be implemented as part of the capstone timeline?
Which requirements will be implemented in the future?}

\newpage{}

\section*{Appendix --- Reflection}

\wss{Not required for CAS 741}

The purpose of reflection questions is to give you a chance to assess your own
learning and that of your group as a whole, and to find ways to improve in the
future. Reflection is an important part of the learning process.  Reflection is
also an essential component of a successful software development process.  

Reflections are most interesting and useful when they're honest, even if the
stories they tell are imperfect. You will be marked based on your depth of
thought and analysis, and not based on the content of the reflections
themselves. Thus, for full marks we encourage you to answer openly and honestly
and to avoid simply writing ``what you think the evaluator wants to hear.''

Please answer the following questions.  Some questions can be answered on the
team level, but where appropriate, each team member should write their own
response:


\begin{enumerate}
    \item What went well while writing this deliverable? 
    \item What pain points did you experience during this deliverable, and how
    did you resolve them?
    \item Which of your listed risks had your team thought of before this
    deliverable, and which did you think of while doing this deliverable? For
    the latter ones (ones you thought of while doing the Hazard Analysis), how
    did they come about?
    \item Other than the risk of physical harm (some projects may not have any
    appreciable risks of this form), list at least 2 other types of risk in
    software products. Why are they important to consider?
\end{enumerate}

\end{document}