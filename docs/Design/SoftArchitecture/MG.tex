\documentclass[12pt, titlepage]{article}

\usepackage{fullpage}
\usepackage[round]{natbib}
\usepackage{multirow}
\usepackage{booktabs}
\usepackage{tabularx}
\usepackage{graphicx}
\usepackage{float}
\usepackage{hyperref}
\usepackage{array}

\hypersetup{
    colorlinks,
    citecolor=blue,
    filecolor=black,
    linkcolor=red,
    urlcolor=blue
}

\input{../../Comments}
%% Common Parts

\newcommand{\progname}{Software Engineering} % PUT YOUR PROGRAM NAME HERE
\newcommand{\authname}{Team 13, Speech Buddies
\\ Mazen Youssef
\\ Rawan Mahdi
\\ Luna Aljammal
\\ Kelvin Yu} % AUTHOR NAMES                  

\usepackage{hyperref}
    \hypersetup{colorlinks=true, linkcolor=blue, citecolor=blue, filecolor=blue,
                urlcolor=blue, unicode=false}
    \urlstyle{same}
                                


\newcounter{acnum}
\newcommand{\actheacnum}{AC\theacnum}
\newcommand{\acref}[1]{AC\ref{#1}}

\newcounter{ucnum}
\newcommand{\uctheucnum}{UC\theucnum}
\newcommand{\uref}[1]{UC\ref{#1}}

\newcounter{mnum}
\newcommand{\mthemnum}{M\themnum}
\newcommand{\mref}[1]{M\ref{#1}}

\begin{document}

\title{Module Guide for Software Engineering} 
\author{\authname}
\date{\today}

\maketitle

\pagenumbering{roman}

\section{Revision History}

\begin{tabularx}{\textwidth}{p{3cm}p{2cm}X}
\toprule {\bf Date} & {\bf Version} & {\bf Notes}\\
\midrule
Nov 11, 2025 & 1.0 & Added modules and Traceability Matrix\\
Nov 12, 2025 & 1.0 & Added Modules M6-M11\\
Nov 12, 2025 & 1.0 & Added Intro, Hierarchy Table, AC and UC\\

\bottomrule
\end{tabularx}

\newpage

\section{Reference Material}

This section records information for easy reference.

\subsection{Abbreviations and Acronyms}

\renewcommand{\arraystretch}{1.2}
\begin{tabular}{l l} 
  \toprule		
  \textbf{symbol} & \textbf{description}\\
  \midrule 
  AC & Anticipated Change\\
  DAG & Directed Acyclic Graph \\
  M & Module \\
  MG & Module Guide \\
  OS & Operating System \\
  R & Requirement\\
  SC & Scientific Computing \\
  SRS & Software Requirements Specification\\
  \progname & Explanation of program name\\
  UC & Unlikely Change \\
  FR & Functional Requirement\\
  LF & Look and Feel \\
  PF & Performance \\
  OER & Operational \& Environmental \\
  MS & Maintainability \& Support \\
  SEC & Security \\
  CUL & Cultural \\
  CPL & Compliance \\
  \bottomrule
\end{tabular}\\

\newpage

\tableofcontents

\listoftables

\listoffigures

\newpage

\pagenumbering{arabic}

\section{Introduction}

Decomposing a system into modules is a commonly accepted approach to developing
software.  A module is a work assignment for a programmer or programming
team~\citep{ParnasEtAl1984}.   For VoiceBridge, we adopt this decomposition based fundamentally on Parnas’s principle of information hiding~\citep{Parnas1972a}.  This
principle supports design for change, because the ``secrets'' that each module
hides represent likely future changes.  This approach supports design for change, a critical concern in assistive technologies where adaptation to evolving user needs and cutting-edge speech recognition improvements frequently occur, especially in early development phases.

Our design follows the rules layed out by \citet{ParnasEtAl1984}, as follows:
\begin{itemize}
\item System details that are likely to change independently should be the
  secrets of separate modules.
\item Each data structure is implemented in only one module.
\item Any other program that requires information stored in a module's data
  structures must obtain it by calling access programs belonging to that module.
\end{itemize}

After completing the first stage of the design, the Software Requirements
Specification (SRS), the Module Guide (MG) is developed~\citep{ParnasEtAl1984}. The MG
specifies the modular structure of the system and is intended to allow both
designers and maintainers to easily identify the parts of the software.  The
potential readers of this document are as follows:

\begin{itemize}
\item New project members: This document can be a guide for a new project member
  to easily understand the overall structure and quickly find the
  relevant modules they are searching for.
\item Maintainers: The hierarchical structure of the module guide improves the
  maintainers' understanding when they need to make changes to the system. It is
  important for a maintainer to update the relevant sections of the document
  after changes have been made.
\item Designers: Once the module guide has been written, it can be used to
  check for consistency, feasibility, and flexibility. Designers can verify the
  system in various ways, such as consistency among modules, feasibility of the
  decomposition, and flexibility of the design.
\end{itemize}

The rest of the document is organized as follows. Section
\ref{SecChange} lists the anticipated and unlikely changes of the software
requirements. Section \ref{SecMH} summarizes the module decomposition that
was constructed according to the likely changes. Section \ref{SecConnection}
specifies the connections between the software requirements and the
modules. Section \ref{SecMD} gives a detailed description of the
modules. Section \ref{SecTM} includes two traceability matrices. One checks
the completeness of the design against the requirements provided in the SRS. The
other shows the relation between anticipated changes and the modules. Section
\ref{SecUse} describes the use relation between modules.

\section{Anticipated and Unlikely Changes} \label{SecChange}

This section lists possible changes to the system. According to the likeliness
of the change, the possible changes are classified into two
categories. Anticipated changes are listed in Section \ref{SecAchange}, and
unlikely changes are listed in Section \ref{SecUchange}.

\subsection{Anticipated Changes} \label{SecAchange}

Anticipated changes are the source of the information that is to be hidden
inside the modules. Ideally, changing one of the anticipated changes will only
require changing the one module that hides the associated decision. The approach
adapted here is called design for
change.

\begin{description}

\item[\refstepcounter{acnum} \actheacnum \label{acHardware}:]
Hardware platform and input devices. \\
Advancements in microphone technology, such as arrays or new form factors, alongside evolving processing hardware, may improve audio quality, reduce costs, and enable integration with wearable devices. These changes influence audio acquisition and preprocessing modules.

\item[\refstepcounter{acnum} \actheacnum \label{acSpeechModel}:]
Speech-to-text engine and language support. \\
The system might expand to support additional languages and dialects beyond English requiring adaptations in ASR models and intent interpretation logic. Continuous improvements in speech recognition accuracy and noise filtering will also necessitate regular updates.

\item[\refstepcounter{acnum} \actheacnum \label{acUserPersonalization}:]
User profile and personalization management. \\
Personalization features may evolve to accommodate changing user speech patterns and preferences, demanding updates in model fine-tuning, adaptive prompting based on confidence, and profile management strategies.

\item[\refstepcounter{acnum} \actheacnum \label{acCommandSupport}:]
Command mapping and browser automation protocols. \\
Updates may introduce new commands, integrate with emerging browser APIs, or improve error detection and recovery in command execution modules.

\item[\refstepcounter{acnum} \actheacnum \label{acUIAndAccessibility}:]
User interface and accessibility compliance. \\
Responsive adjustments to UI designs and accessibility layers will be required to stay compliant with evolving WCAG guidelines and to address user feedback for improved usability.

\item[\refstepcounter{acnum} \actheacnum \label{acErrorHandling}:]
Error handling and recovery policies. \\
Classification schemes, messaging protocols, retry/backoff strategies, and compensation mechanisms may be enhanced to increase robustness and user experience quality.

\item[\refstepcounter{acnum} \actheacnum \label{acSessionManagement}:]
Session lifecycle and interaction flow. \\
Changes may extend session duration limits, improve state persistence, and better handle asynchronous or interrupted user interactions.

\item[\refstepcounter{acnum} \actheacnum \label{acDataPrivacyLogging}:]
Data privacy, audit logging, and security. \\
Adaptations will be necessary to comply with evolving data privacy laws and security best practices, affecting encryption, audit trail formats, and consent mechanisms.

\item[\refstepcounter{acnum} \actheacnum \label{acPersonalizationLayer}:]
Personalization and prompting enhancements. \\
Enhancements aiming at more context-aware and confidence-driven user assistance will continually refine prompting and instruction modules.

\end{description}


% \wss{Anticipated changes relate to changes that would be made in requirements,
% design or implementation choices.  They are not related to changes that are made
% at run-time, like the values of parameters.}

\subsection{Unlikely Changes} \label{SecUchange}

The module design should be as general as possible. However, a general system is
more complex. Sometimes this complexity is not necessary. Fixing some design
decisions at the system architecture stage can simplify the software design. If
these decision should later need to be changed, then many parts of the design
will potentially need to be modified. Hence, it is not intended that these
decisions will be changed.

\begin{description}

\item[\refstepcounter{ucnum} \uctheucnum \label{ucHardwarePlatform}:]
Switching to specialized or high-end microphones and audio hardware. \\
VoiceBridge targets consumer-grade devices using standard microphones, ensuring affordability and accessibility. Moving to uncommon or costly hardware is outside the project scope and is therefore unlikely.

\item[\refstepcounter{ucnum} \uctheucnum \label{ucProcessingOffload}:]
Offloading speech processing or command execution to external or cloud systems. \\
To protect user privacy and maintain responsiveness, VoiceBridge performs all key processing locally. Changing this would require major architecture redesign and is not planned.

\item[\refstepcounter{ucnum} \uctheucnum \label{ucASRMethod}:]
Fundamental changes to the speech recognition approach. \\
The system uses specialized ASR models tuned for speech impairments based on proven models. Major shifts in ASR methodology would be disruptive and are therefore improbable.

\item[\refstepcounter{ucnum} \uctheucnum \label{ucCommandSystem}:]
Complete redesign of command mapping and browser control modules. \\
These modules are tightly coupled with browser APIs and OS functions for stability and usability. Major rewrites would cause extensive system impact and are unlikely.

\item[\refstepcounter{ucnum} \uctheucnum \label{ucArchitecture}:]
Abandoning modular design for a monolithic or radically different architecture. \\
Modularity was chosen for maintainability and scalability. Changing to a less modular approach would increase complexity and reduce flexibility.

\end{description}

\section{Module Hierarchy} \label{SecMH}

This section provides an overview of the module design. The system is organized into architectural layers, each containing modules that encapsulate specific responsibilities. Table~\ref{TblMH} presents the hierarchy of modules by their layers, including their primary purposes and responsibilities. The modules listed below, which are leaves in the hierarchy tree, are the modules that will be implemented.

\begin{description}
\item [\refstepcounter{mnum} \mthemnum \label{m1}:] User Interface Module
\item [\refstepcounter{mnum} \mthemnum \label{m2}:] Accessibility Layer
\item [\refstepcounter{mnum} \mthemnum \label{m3}:] Feedback Display Module
\item [\refstepcounter{mnum} \mthemnum \label{m4}:] Speech-to-Text Engine
\item [\refstepcounter{mnum} \mthemnum \label{m5}:] Intent Interpreter
\item [\refstepcounter{mnum} \mthemnum \label{m6}:] Command Mapping Module
\item [\refstepcounter{mnum} \mthemnum \label{m7}:] Command Execution Layer
\item [\refstepcounter{mnum} \mthemnum \label{m8}:] Error Feedback
\item [\refstepcounter{mnum} \mthemnum \label{m9}:] Browser Controller
\item [\refstepcounter{mnum} \mthemnum \label{m10}:] Session Manager
\item [\refstepcounter{mnum} \mthemnum \label{m11}:] Error Handling \& Recovery Module
\item [\refstepcounter{mnum} \mthemnum \label{m12}:] Data Management Layer
\item [\refstepcounter{mnum} \mthemnum \label{m13}:] User Profile Manager
\item [\refstepcounter{mnum} \mthemnum \label{m14}:] Audit Logger
\item [\refstepcounter{mnum} \mthemnum \label{m15}:] Credential Manager
\item [\refstepcounter{mnum} \mthemnum \label{m16}:] Encryption Manager
\item [\refstepcounter{mnum} \mthemnum \label{m17}:] Out-of-Scope Handler
\item [\refstepcounter{mnum} \mthemnum \label{m18}:] Microphone Manager
\item [\refstepcounter{mnum} \mthemnum \label{m19}:] VAD Noise Filter
\item [\refstepcounter{mnum} \mthemnum \label{m20}:] Prompting Module
\item [\refstepcounter{mnum} \mthemnum \label{m21}:] Model Tuner
\item [\refstepcounter{mnum} \mthemnum \label{m22}:] Instruction Registry
\end{description}

\section{Connection Between Requirements and Design}

The design of the system is intended to satisfy the requirements developed in the 
(\href{https://github.com/speech-buddies/VoiceBridge/blob/main/docs/SRS-Volere/SRS.pdf}{SRS}).
In this stage, the system is decomposed into modules. The connection between requirements and modules is listed in Table~\ref{TblMH}.

\begin{table}[htbp]
\centering
\small
\renewcommand{\arraystretch}{1.1} % row height
\setlength{\tabcolsep}{6pt}       % column spacing
\begin{tabular}{
    >{\raggedright\arraybackslash}p{3cm}
    >{\raggedright\arraybackslash}p{4.8cm}
    >{\raggedright\arraybackslash}p{8.2cm}
}

\toprule
\textbf{Architectural Layer} & \textbf{Included Modules} & \textbf{Purpose / Responsibilities} \\
\midrule

\textbf{Presentation Layer} &
User Interface Module \newline
Accessibility Layer \newline
Feedback Display Module &
Manages direct user interaction and feedback presentation. Displays transcribed text, confirmations, feedback messages, and status indicators. Ensures compliance with WCAG accessibility standards including contrast, font, and color. \\
\midrule

\textbf{Application (Processing) Layer} &
Speech-to-Text Engine \newline
Intent Interpreter \newline
Command Mapping Module \newline
Command Execution Layer \newline
Error Feedback Module &
Performs speech recognition and intent processing. Converts speech audio to text, interprets user intent, maps intent to browser or OS commands, executes commands, and handles error feedback. \\
\midrule

\textbf{Control (Orchestration) Layer} &
Browser Controller \newline
Session Manager \newline
Error Handling and Recovery Module &
Coordinates and manages flow between modules. Tracks session states such as capture, transcribe, confirm, and execute. Manages events, exceptions, retries, and cancellations to ensure smooth operation. \\
\midrule

\textbf{Data Management Layer} &
Data Storage Manager \newline
User Profile Manager \newline
Audit Logger &
Maintains persistent data and user personalization. Stores user speech samples, preferences, command mappings, transcripts, and logs. Supports diagnostics and evaluation. \\
\midrule

\textbf{Security Layer} &
Credential Manager \newline
Encryption Manager \newline
Out-of-Scope Handler &
Manages authentication, encryption, and system boundaries. Ensures secure integration with external APIs and operating system interfaces. Handles login confirmations. \\
\midrule

\textbf{Input Processing Layer} &
Microphone Manager \newline
VAD and Noise Filter &
Controls microphone hardware and audio capture. Filters noise to improve voice input quality. \\
\midrule

\textbf{Personalization Layer} &
Prompting Module \newline
Model Tuner \newline
Instruction Registry &
Provides user prompting, model fine-tuning, and instruction management. Supports optional personalization based on confidence scores and usage history. \\

\bottomrule
\end{tabular}
\caption{Module Hierarchy}
\label{TblMH}
\end{table}


% =========================
% Module Decomposition
% =========================
\section{Module Decomposition}

% ---------- Application (Processing) ----------
\subsection*{Application (Processing) Modules}

\subsubsection*{M7: Command Execution Layer}
\textbf{Secrets:} How commands are validated, mapped to backend requests, dispatched, tracked, cancelled, and timed-out. Internal queueing and audit format. \\
\textbf{Services:} Execute/cancel/status for commands; translate to backend format; rollback hooks. \\
\textbf{Implemented By:} Core runtime (browser bridge + OS process APIs). \\

\medskip
\subsubsection*{M8: Error Feedback}
\textbf{Secrets:} Message templating, localization hooks, mapping of internal error classes to user-facing copy and recovery options. \\
\textbf{Services:} Show error; show recovery prompt with options; dismiss item; log event. \\
\textbf{Implemented By:} UI notification layer. \\

% ---------- Control (Orchestration) ----------
\subsection*{Control (Orchestration) Modules}

\subsubsection*{M9: BrowserController}
\textbf{Secrets:} Transport protocol to the browser controller, timeout wrappers, normalization of responses, and session plumbing. \\
\textbf{Services:} Send request; get status; cancel; open/close controller session. \\
\textbf{Implemented By:} Automation bridge client. \\

\medskip
\subsubsection*{M10: Session Manager}
\textbf{Secrets:} Session lifecycle rules, storage schema, and TTL/expiry handling. \\
\textbf{Services:} Start/stop session; get session state; attach command; set state. \\
\textbf{Implemented By:} In-memory cache + persistent store. \\

\medskip
\subsubsection*{M11: Error Handling \& Recovery Module}
\textbf{Secrets:} Error classification policy, retry/backoff strategy, and compensation catalog. \\
\textbf{Services:} Handle error; retry; compensate; classify; record event. \\
\textbf{Implemented By:} Orchestration runtime with policy store. \\


\section{Connection Between Requirements and Design} \label{SecConnection}

The design of the system is intended to satisfy the requirements developed in
the SRS. In this stage, the system is decomposed into modules. The connection
between requirements and modules is listed in Table~\ref{TblRT}.

\wss{The intention of this section is to document decisions that are made
  ``between'' the requirements and the design.  To satisfy some requirements,
  design decisions need to be made.  Rather than make these decisions implicit,
  they are explicitly recorded here.  For instance, if a program has security
  requirements, a specific design decision may be made to satisfy those
  requirements with a password.}

\section{Module Decomposition} \label{SecMD}

Modules are decomposed according to the principle of ``information hiding''
proposed by \citet{ParnasEtAl1984}. The \emph{Secrets} field in a module
decomposition is a brief statement of the design decision hidden by the
module. The \emph{Services} field specifies \emph{what} the module will do
without documenting \emph{how} to do it. For each module, a suggestion for the
implementing software is given under the \emph{Implemented By} title. If the
entry is \emph{OS}, this means that the module is provided by the operating
system or by standard programming language libraries.  \emph{\progname{}} means the
module will be implemented by the \progname{} software.

Only the leaf modules in the hierarchy have to be implemented. If a dash
(\emph{--}) is shown, this means that the module is not a leaf and will not have
to be implemented.

\subsection{Hardware Hiding Modules (\mref{mHH})}

\begin{description}
\item[Secrets:]The data structure and algorithm used to implement the virtual
  hardware.
\item[Services:]Serves as a virtual hardware used by the rest of the
  system. This module provides the interface between the hardware and the
  software. So, the system can use it to display outputs or to accept inputs.
\item[Implemented By:] OS
\end{description}

\subsection{Behaviour-Hiding Module}

\begin{description}
\item[Secrets:]The contents of the required behaviours.
\item[Services:]Includes programs that provide externally visible behaviour of
  the system as specified in the software requirements specification (SRS)
  documents. This module serves as a communication layer between the
  hardware-hiding module and the software decision module. The programs in this
  module will need to change if there are changes in the SRS.
\item[Implemented By:] --
\end{description}

\subsubsection{Input Format Module (\mref{mInput})}

\begin{description}
\item[Secrets:]The format and structure of the input data.
\item[Services:]Converts the input data into the data structure used by the
  input parameters module.
\item[Implemented By:] [Your Program Name Here]
\item[Type of Module:] [Record, Library, Abstract Object, or Abstract Data Type]
  [Information to include for leaf modules in the decomposition by secrets tree.]
\end{description}

\subsubsection{Etc.}


\subsection{Software Decision Module}

\begin{description}
\item[Secrets:] The design decision based on mathematical theorems, physical
  facts, or programming considerations. The secrets of this module are
  \emph{not} described in the SRS.
\item[Services:] Includes data structure and algorithms used in the system that
  do not provide direct interaction with the user. 
  % Changes in these modules are more likely to be motivated by a desire to
  % improve performance than by externally imposed changes.
\item[Implemented By:] --
\end{description}

\subsubsection{Etc.}

\section{Traceability Matrix} \label{SecTM}

This section shows two traceability matrices: between the modules and the
requirements and between the modules and the anticipated changes.

% the table should use mref, the requirements should be named, use something
% like fref
\begin{table}[H]
\centering
\begin{tabular}{p{0.2\textwidth} p{0.6\textwidth}}
\toprule
\textbf{Req.} & \textbf{Modules}\\
\midrule
FR-1 10.1 & \mref{m18}, \mref{m19}, \mref{m4}, \mref{m10}\\
FR-2 10.2& \mref{m4}, \mref{m19}, \mref{m21}, \mref{m12}\\
FR-3 10.3& \mref{m1}, \mref{m2}, \mref{m3}, \mref{m8}\\
FR-4 10.4& \mref{m5}, \mref{m6}, \mref{m22}\\
FR-5 10.5& \mref{m7}, \mref{m9}, \mref{m11}, \mref{m8}\\
\bottomrule
\end{tabular}
\caption{Trace Between Functional Requirements and Modules}
\label{TblRTFunctional}
\end{table}

\begin{table}[H]
\centering
\begin{tabular}{p{0.2\textwidth} p{0.6\textwidth}}
\toprule
\textbf{Req.} & \textbf{Modules}\\
\midrule
LF-1 11.1& \mref{m1}, \mref{m2}, \mref{m3}, \mref{m8}\\
LF-2 11.2& \mref{m1}, \mref{m2}, \mref{m20}, \mref{m3}\\
\bottomrule
\end{tabular}
\caption{Trace Between Look \& Feel Requirements and Modules}
\label{TblRTLF}
\end{table}

\begin{table}[H]
\centering
\begin{tabular}{p{0.2\textwidth} p{0.6\textwidth}}
\toprule
\textbf{Req.} & \textbf{Modules}\\
\midrule
UH-1 12.1 & \mref{m1}, \mref{m2}, \mref{m3}, \mref{m8}\\
UH-2 12.2 & \mref{m13}, \mref{m12}, \mref{m5}, \mref{m6}\\
UH-3 12.3& \mref{m1}, \mref{m2}, \mref{m20}, \mref{m10}\\
UH-4 12.4& \mref{m8}, \mref{m3}, \mref{m2}, \mref{m20}\\
UH-5 12.5& \mref{m2}, \mref{m1}, \mref{m3}, \mref{m8}\\
\bottomrule
\end{tabular}
\caption{Trace Between Usability \& Humanity Requirements and Modules}
\label{TblRTUsabilityHumanity}
\end{table}

\begin{table}[H]
\centering
\begin{tabular}{p{0.2\textwidth} p{0.6\textwidth}}
\toprule
\textbf{Req.} & \textbf{Modules}\\
\midrule
PF-1 13.1 & \mref{m4}, \mref{m5}, \mref{m6}, \mref{m7}, \mref{m9}\\
PF-2 13.2& \mref{m5}, \mref{m6}, \mref{m7}, \mref{m8}, \mref{m11}\\
PF-3 13.3& \mref{m4}, \mref{m19}, \mref{m21}, \mref{m5}\\
PF-4 13.4& \mref{m11}, \mref{m10}, \mref{m8}, \mref{m9}\\
PF-5 13.5 & \mref{m10}, \mref{m12}, \mref{m13}\\
PF-6 13.6& \mref{m10}, \mref{m12}, \mref{m7}, \mref{m9}\\
PF-7 13.7& \mref{m12}, \mref{m13}, \mref{m14}, \mref{m15}\\
\bottomrule
\end{tabular}
\caption{Trace Between Performance Requirements and Modules}
\label{TblRTPerformance}
\end{table}

\begin{table}[H]
\centering
\begin{tabular}{p{0.2\textwidth} p{0.6\textwidth}}
\toprule
\textbf{Req.} & \textbf{Modules}\\
\midrule
OER-1 14.1& \mref{m18}, \mref{m19}, \mref{m4}, \mref{m11}\\
OER-2 14.2& \mref{m1}, \mref{m2}, \mref{m9}, \mref{m10}\\
OER-3 14.3& \mref{m5}, \mref{m6}, \mref{m9}, \mref{m22}\\
OER-4 14.4& \mref{m12}, \mref{m13}, \mref{m14}, \mref{m15}, \mref{m10}\\
OER-5 14.5& \mref{m10}, \mref{m12}, \mref{m14}, \mref{m15}\\
\bottomrule
\end{tabular}
\caption{Trace Between Operational \& Environmental Requirements and Modules}
\label{TblRTOperationalEnv}
\end{table}

\begin{table}[H]
\centering
\begin{tabular}{p{0.2\textwidth} p{0.6\textwidth}}
\toprule
\textbf{Req.} & \textbf{Modules}\\
\midrule
MS-1 15.1& \mref{m10}, \mref{m12}, \mref{m13}, \mref{m14}, \mref{m15}, \mref{m22}\\
MS-2 15.2& \mref{m1}, \mref{m2}, \mref{m3}, \mref{m8}, \mref{m20}\\
MS-3 15.3& \mref{m1}, \mref{m2}, \mref{m9}, \mref{m10}, \mref{m11}\\
\bottomrule
\end{tabular}
\caption{Trace Between Maintainability \& Support Requirements and Modules}
\label{TblRTMaintainSupport}
\end{table}

\begin{table}[H]
\centering
\begin{tabular}{p{0.2\textwidth} p{0.6\textwidth}}
\toprule
\textbf{Req.} & \textbf{Modules}\\
\midrule
SEC-1 16.1 & \mref{m15}, \mref{m16}, \mref{m13}, \mref{m12}, \mref{m10}\\
SEC-2 16.2& \mref{m5}, \mref{m6}, \mref{m7}, \mref{m8}, \mref{m11}\\
SEC-3 16.3& \mref{m12}, \mref{m13}, \mref{m14}, \mref{m15}, \mref{m16}\\
SEC-4 16.4& \mref{m14}, \mref{m12}, \mref{m10}, \mref{m15}\\
SEC-5 16.5& \mref{m11}, \mref{m8}, \mref{m19}, \mref{m7}, \mref{m9}\\
\bottomrule
\end{tabular}
\caption{Trace Between Security Requirements and Modules}
\label{TblRTSecurity}
\end{table}

\begin{table}[H]
\centering
\begin{tabular}{p{0.2\textwidth} p{0.6\textwidth}}
\toprule
\textbf{Req.} & \textbf{Modules}\\
\midrule
CUL-1 17.1& \mref{m2}, \mref{m3}, \mref{m8}, \mref{m20}, \mref{m5}\\
\bottomrule
\end{tabular}
\caption{Trace Between Cultural Requirements and Modules}
\label{TblRTCultural}
\end{table}

\begin{table}[H]
\centering
\begin{tabular}{p{0.2\textwidth} p{0.6\textwidth}}
\toprule
\textbf{Req.} & \textbf{Modules}\\
\midrule
CPL-1 18.1& \mref{m12}, \mref{m13}, \mref{m14}, \mref{m15}, \mref{m16}\\
CPL-2 18.2& \mref{m1}, \mref{m2}, \mref{m3}, \mref{m8}\\
\bottomrule
\end{tabular}
\caption{Trace Between Compliance Requirements and Modules}
\label{TblRTCompliance}
\end{table}








\section{Use Hierarchy Between Modules} \label{SecUse}

In this section, the uses hierarchy between modules is
provided. \citet{Parnas1978} said of two programs A and B that A {\em uses} B if
correct execution of B may be necessary for A to complete the task described in
its specification. That is, A {\em uses} B if there exist situations in which
the correct functioning of A depends upon the availability of a correct
implementation of B.  Figure \ref{FigUH} illustrates the use relation between
the modules. It can be seen that the graph is a directed acyclic graph
(DAG). Each level of the hierarchy offers a testable and usable subset of the
system, and modules in the higher level of the hierarchy are essentially simpler
because they use modules from the lower levels.

\wss{The uses relation is not a data flow diagram.  In the code there will often
be an import statement in module A when it directly uses module B.  Module B
provides the services that module A needs.  The code for module A needs to be
able to see these services (hence the import statement).  Since the uses
relation is transitive, there is a use relation without an import, but the
arrows in the diagram typically correspond to the presence of import statement.}

\wss{If module A uses module B, the arrow is directed from A to B.}

\begin{figure}[H]
\centering
%\includegraphics[width=0.7\textwidth]{UsesHierarchy.png}
\caption{Use hierarchy among modules}
\label{FigUH}
\end{figure}

%\section*{References}

\section{User Interfaces}

\wss{Design of user interface for software and hardware.  Attach an appendix if
needed. Drawings, Sketches, Figma}

\section{Design of Communication Protocols}

\wss{If appropriate}

\section{Timeline}

\wss{Schedule of tasks and who is responsible}

\wss{You can point to GitHub if this information is included there}

\bibliographystyle {plainnat}
\bibliography{../../../refs/References}

\newpage{}

\end{document}