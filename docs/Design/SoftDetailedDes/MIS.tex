\documentclass[12pt, titlepage]{article}

\usepackage{amsmath, mathtools}

\usepackage[round]{natbib}
\usepackage{amsfonts}
\usepackage{amssymb}
\usepackage{graphicx}
\usepackage{colortbl}
\usepackage{xr}
\usepackage{hyperref}
\usepackage{longtable}
\usepackage{xfrac}
\usepackage{tabularx}
\usepackage{float}
\usepackage{siunitx}
\usepackage{booktabs}
\usepackage{multirow}
\usepackage[section]{placeins}
\usepackage{caption}
\usepackage{fullpage}

\hypersetup{
bookmarks=true,     % show bookmarks bar?
colorlinks=true,       % false: boxed links; true: colored links
linkcolor=red,          % color of internal links (change box color with linkbordercolor)
citecolor=blue,      % color of links to bibliography
filecolor=magenta,  % color of file links
urlcolor=cyan          % color of external links
}

\usepackage{array}

\externaldocument{../../SRS/SRS}

\input{../../Comments}
%% Common Parts

\newcommand{\progname}{Software Engineering} % PUT YOUR PROGRAM NAME HERE
\newcommand{\authname}{Team 13, Speech Buddies
\\ Mazen Youssef
\\ Rawan Mahdi
\\ Luna Aljammal
\\ Kelvin Yu} % AUTHOR NAMES                  

\usepackage{hyperref}
    \hypersetup{colorlinks=true, linkcolor=blue, citecolor=blue, filecolor=blue,
                urlcolor=blue, unicode=false}
    \urlstyle{same}
                                


\begin{document}

\title{Module Interface Specification for \progname{}}

\author{\authname}

\date{\today}

\maketitle

\pagenumbering{roman}

\section{Revision History}

\begin{tabularx}{\textwidth}{p{3cm}p{2cm}X}
\toprule {\bf Date} & {\bf Version} & {\bf Notes}\\
\midrule
Date 1 & 1.0 & Notes\\
Date 2 & 1.1 & Notes\\
\bottomrule
\end{tabularx}

~\newpage

\section{Symbols, Abbreviations and Acronyms}

See SRS Documentation at \wss{give url}

\wss{Also add any additional symbols, abbreviations or acronyms}

\newpage

\tableofcontents

\newpage

\pagenumbering{arabic}

\section{Introduction}

The following document details the Module Interface Specifications for
\wss{Fill in your project name and description}

Complementary documents include the System Requirement Specifications
and Module Guide.  The full documentation and implementation can be
found at \url{...}.  \wss{provide the url for your repo}

\section{Notation}

\wss{You should describe your notation.  You can use what is below as
  a starting point.}

The structure of the MIS for modules comes from \citet{HoffmanAndStrooper1995},
with the addition that template modules have been adapted from
\cite{GhezziEtAl2003}.  The mathematical notation comes from Chapter 3 of
\citet{HoffmanAndStrooper1995}.  For instance, the symbol := is used for a
multiple assignment statement and conditional rules follow the form $(c_1
\Rightarrow r_1 | c_2 \Rightarrow r_2 | ... | c_n \Rightarrow r_n )$.

The following table summarizes the primitive data types used by \progname. 

\begin{center}
\renewcommand{\arraystretch}{1.2}
\noindent 
\begin{tabular}{l l p{7.5cm}} 
\toprule 
\textbf{Data Type} & \textbf{Notation} & \textbf{Description}\\ 
\midrule
character & char & a single symbol or digit\\
integer & $\mathbb{Z}$ & a number without a fractional component in (-$\infty$, $\infty$) \\
natural number & $\mathbb{N}$ & a number without a fractional component in [1, $\infty$) \\
real & $\mathbb{R}$ & any number in (-$\infty$, $\infty$)\\
\bottomrule
\end{tabular} 
\end{center}

\noindent
The specification of \progname \ uses some derived data types: sequences, strings, and
tuples. Sequences are lists filled with elements of the same data type. Strings
are sequences of characters. Tuples contain a list of values, potentially of
different types. In addition, \progname \ uses functions, which
are defined by the data types of their inputs and outputs. Local functions are
described by giving their type signature followed by their specification.

\section{Module Decomposition}

The following table is taken directly from the Module Guide document for this project.

\begin{table}[h!]
\centering
\begin{tabular}{p{0.3\textwidth} p{0.6\textwidth}}
\toprule
\textbf{Level 1} & \textbf{Level 2}\\
\midrule

{Hardware-Hiding} & ~ \\
\midrule

\multirow{7}{0.3\textwidth}{Behaviour-Hiding} & Input Parameters\\
& Output Format\\
& Output Verification\\
& Temperature ODEs\\
& Energy Equations\\ 
& Control Module\\
& Specification Parameters Module\\
\midrule

\multirow{3}{0.3\textwidth}{Software Decision} & {Sequence Data Structure}\\
& ODE Solver\\
& Plotting\\
\bottomrule

\end{tabular}
\caption{Module Hierarchy}
\label{TblMH}
\end{table}

\newpage
~\newpage

\section{MIS of \wss{Module Name}} \label{Module} \wss{Use labels for
  cross-referencing}

\wss{You can reference SRS labels, such as R\ref{R_Inputs}.}

\wss{It is also possible to use \LaTeX for hypperlinks to external documents.}

\subsection{Module}

\wss{Short name for the module}

\subsection{Uses}


\subsection{Syntax}

\subsubsection{Exported Constants}

\subsubsection{Exported Access Programs}

\begin{center}
\begin{tabular}{p{2cm} p{4cm} p{4cm} p{2cm}}
\hline
\textbf{Name} & \textbf{In} & \textbf{Out} & \textbf{Exceptions} \\
\hline
\wss{accessProg} & - & - & - \\
\hline
\end{tabular}
\end{center}

\subsection{Semantics}

\subsubsection{State Variables}

\wss{Not all modules will have state variables.  State variables give the module
  a memory.}

\subsubsection{Environment Variables}

\wss{This section is not necessary for all modules.  Its purpose is to capture
  when the module has external interaction with the environment, such as for a
  device driver, screen interface, keyboard, file, etc.}

\subsubsection{Assumptions}

\wss{Try to minimize assumptions and anticipate programmer errors via
  exceptions, but for practical purposes assumptions are sometimes appropriate.}

\subsubsection{Access Routine Semantics}

\noindent \wss{accessProg}():
\begin{itemize}
\item transition: \wss{if appropriate} 
\item output: \wss{if appropriate} 
\item exception: \wss{if appropriate} 
\end{itemize}

\wss{A module without environment variables or state variables is unlikely to
  have a state transition.  In this case a state transition can only occur if
  the module is changing the state of another module.}

\wss{Modules rarely have both a transition and an output.  In most cases you
  will have one or the other.}

\subsubsection{Local Functions}

\wss{As appropriate} \wss{These functions are for the purpose of specification.
  They are not necessarily something that is going to be implemented
  explicitly.  Even if they are implemented, they are not exported; they only
  have local scope.}

\newpage
\section{MIS of Command Execution Layer} \label{M7}

\subsection{Module}

Executes verified, mapped commands on the host device via the browser control bridge.
Responsible for dispatch, cancellation, basic guardrails, timeout handling, and result reporting.

\subsection{Uses}

Browser control/automation client (e.g., \texttt{BrowserUse} bridge); OS process APIs; timer utilities;
audit logger; configuration store for timeouts and retries.

\subsection{Syntax}

\subsubsection{Exported Constants}

None.

\subsubsection{Exported Access Programs}

\begin{center}
\begin{tabular}{p{4cm} p{4cm} p{3cm} p{4cm}}
\hline
\textbf{Name} & \textbf{In} & \textbf{Out} & \textbf{Exceptions} \\
\hline
\texttt{CommandExecLayer} & \texttt{backend: Client}, \texttt{timeout\_s: int} & self & - \\
\texttt{execute} & \texttt{cmd: ExecCmd} & \texttt{Result} & \texttt{PermissionError}, \texttt{TimeoutError}, \texttt{ExecError} \\
\texttt{cancel} & \texttt{cmd\_id: UUID} & \texttt{bool} & \texttt{NotFoundError} \\
\texttt{status} & \texttt{cmd\_id: UUID} & \texttt{ExecStatus} & \texttt{NotFoundError} \\
\texttt{validate} & \texttt{cmd: ExecCmd} & \texttt{bool} & - \\
\texttt{map\_to\_backend} & \texttt{cmd: ExecCmd} & \texttt{BackendReq} & \texttt{MappingError} \\
\texttt{rollback} & \texttt{cmd\_id: UUID} & \texttt{bool} & \texttt{ExecError} \\
\hline
\end{tabular}
\end{center}

\subsection{Semantics}

\subsubsection{State Variables}

\begin{itemize}
  \item \texttt{pending\_q: map[UUID]$\to$ExecCtx} — commands in flight
  \item \texttt{default\_timeout\_s: int} — global timeout for executions
  \item \texttt{retries: int} — max retry attempts for transient failures
  \item \texttt{backend: Client} — handle to browser automation bridge
  \item \texttt{audit: Logger} — sink for execution logs
\end{itemize}

\subsubsection{Environment Variables}

\begin{itemize}
  \item \texttt{BROWSER\_BRIDGE\_URL} — connection string for automation client
  \item \texttt{EXEC\_HARD\_LIMIT\_S} — absolute upper bound on execution time
  \item \texttt{LOG\_LEVEL} — audit verbosity
\end{itemize}

\subsubsection{Assumptions}

\subsubsection{Access Routine Semantics}

\noindent

\noindent \texttt{CommandExecLayer}(\texttt{backend}, \texttt{timeout\_s}):
\begin{itemize}
\item transition: initialize \texttt{pending\_q}, set \texttt{default\_timeout\_s}, bind \texttt{backend}, configure \texttt{audit}
\item output: initialized instance
\item exception: -
\end{itemize}

\noindent \texttt{execute}(\texttt{cmd}):
\begin{itemize}
\item transition: add to \texttt{pending\_q}; invoke \texttt{map\_to\_backend}; dispatch to \texttt{backend}; update status; remove on completion
\item output: \texttt{Result} with success flag, message, and optional payload
\item exception: \texttt{PermissionError} if disallowed; \texttt{TimeoutError} if exceeds limits; \texttt{ExecError} on backend failure
\end{itemize}

\noindent \texttt{cancel}(\texttt{cmd\_id}):
\begin{itemize}
\item transition: signal cancellation to backend; mark context as cancelled; remove from \texttt{pending\_q}
\item output: \texttt{true} if cancelled; otherwise \texttt{false}
\item exception: \texttt{NotFoundError} if \texttt{cmd\_id} not tracked
\end{itemize}

\noindent \texttt{status}(\texttt{cmd\_id}):
\begin{itemize}
\item transition: none
\item output: \texttt{ExecStatus} in \{\texttt{queued}, \texttt{running}, \texttt{succeeded}, \texttt{failed}, \texttt{cancelled}\}
\item exception: \texttt{NotFoundError} if unknown
\end{itemize}

\noindent \texttt{validate}(\texttt{cmd}):
\begin{itemize}
\item transition: none
\item output: \texttt{true} iff command matches allowed action set and guardrails (e.g., requires-confirmation flags met)
\item exception: -
\end{itemize}

\noindent \texttt{map\_to\_backend}(\texttt{cmd}):
\begin{itemize}
\item transition: none
\item output: \texttt{BackendReq} (normalized request for the bridge)
\item exception: \texttt{MappingError} if no mapping exists
\end{itemize}

\noindent \texttt{rollback}(\texttt{cmd\_id}):
\begin{itemize}
\item transition: attempt compensating operation (e.g., reopen tab, revert text entry)
\item output: \texttt{true} on success; otherwise \texttt{false}
\item exception: \texttt{ExecError} if rollback fails
\end{itemize}

\subsubsection{Local Functions}

\begin{itemize}
  \item \texttt{start\_timer(ctx)} — begin timeout tracking on a context
  \item \texttt{complete(ctx, result)} — finalize status, log, and cleanup
  \item \texttt{is\_allowed(cmd)} — check guardrails and policy
  \item \texttt{retryable(err)} — decide if backend error can be retried
  \item \texttt{to\_backend(payload)} — format request for bridge
\end{itemize}

\section{MIS of Error Feedback} \label{M8}

\subsection{Module}

Displays user-friendly error messages and recovery prompts. Surfaces execution issues, suggests next steps, and routes critical errors to support logs.

\subsection{Uses}

UI notification layer; audit logger; configuration store for messages; localization service (optional).

\subsection{Syntax}

\subsubsection{Exported Constants}

None.

\subsubsection{Exported Access Programs}

\begin{center}
\begin{tabular}{p{4cm} p{4cm} p{3cm} p{4cm}}
\hline
\textbf{Name} & \textbf{In} & \textbf{Out} & \textbf{Exceptions} \\
\hline
\texttt{ErrorFeedback} & \texttt{notifier: UiClient} & self & - \\
\texttt{show\_error} & \texttt{code: string}, \texttt{detail: string} & - & - \\
\texttt{show\_recovery} & \texttt{cmd\_id: UUID}, \texttt{options: string list} & - & - \\
\texttt{dismiss} & \texttt{feedback\_id: UUID} & \texttt{bool} & - \\
\texttt{log} & \texttt{event: ErrorEvent} & - & - \\
\hline
\end{tabular}
\end{center}

\subsection{Semantics}

\subsubsection{State Variables}

\begin{itemize}
  \item \texttt{active: map[UUID] to FeedbackItem} — currently visible items
  \item \texttt{notifier: UiClient} — handle to UI notifications
\end{itemize}

\subsubsection{Environment Variables}

\begin{itemize}
  \item \texttt{DEFAULT\_LANG} — fallback locale
  \item \texttt{ERROR\_COPY\_PATH} — message templates location
\end{itemize}

\subsubsection{Assumptions}

UI client is available and permitted to display notifications.

\subsubsection{Access Routine Semantics}

\noindent \texttt{ErrorFeedback}(\texttt{notifier}):
\begin{itemize}
\item transition: initialize \texttt{active}; bind \texttt{notifier}
\item output: initialized instance
\item exception: -
\end{itemize}

\noindent \texttt{show\_error}(\texttt{code}, \texttt{detail}):
\begin{itemize}
\item transition: create and register a feedback item in \texttt{active}; display via \texttt{notifier}
\item output: -
\item exception: -
\end{itemize}

\noindent \texttt{show\_recovery}(\texttt{cmd\_id}, \texttt{options}):
\begin{itemize}
\item transition: render recovery prompt with provided options
\item output: -
\item exception: -
\end{itemize}

\noindent \texttt{dismiss}(\texttt{feedback\_id}):
\begin{itemize}
\item transition: remove from \texttt{active}; instruct UI to hide
\item output: \texttt{true} if removed; otherwise \texttt{false}
\item exception: -
\end{itemize}

\noindent \texttt{log}(\texttt{event}):
\begin{itemize}
\item transition: write event to audit log
\item output: -
\item exception: -
\end{itemize}

\subsubsection{Local Functions}

\begin{itemize}
  \item \texttt{format\_message(code, detail)} — produce a concise message
  \item \texttt{make\_recovery(options)} — build prompt content
\end{itemize}

\section{MIS of BrowserController} \label{M9}

\subsection{Module}

Handles interaction with the browser controller: sends backend requests, receives statuses, and streams results back to higher layers.

\subsection{Uses}

Browser automation bridge client; transport layer; timer utilities; audit logger.

\subsection{Syntax}

\subsubsection{Exported Constants}

None.

\subsubsection{Exported Access Programs}

\begin{center}
\begin{tabular}{p{4cm} p{4cm} p{3cm} p{4cm}}
\hline
\textbf{Name} & \textbf{In} & \textbf{Out} & \textbf{Exceptions} \\
\hline
\texttt{BrowserController} & \texttt{bridge: Client}, \texttt{timeout\_s: int} & self & - \\
\texttt{send} & \texttt{req: BackendReq} & \texttt{BackendResp} & \texttt{TimeoutError}, \texttt{TransportError} \\
\texttt{get\_status} & \texttt{cmd\_id: UUID} & \texttt{ExecStatus} & \texttt{NotFoundError} \\
\texttt{cancel} & \texttt{cmd\_id: UUID} & \texttt{bool} & \texttt{NotFoundError} \\
\texttt{open\_session} & \texttt{user\_id: UUID} & \texttt{SessionId} & \texttt{TransportError} \\
\texttt{close\_session} & \texttt{session\_id: SessionId} & \texttt{bool} & \texttt{TransportError} \\
\hline
\end{tabular}
\end{center}

\subsection{Semantics}

\subsubsection{State Variables}

\begin{itemize}
  \item \texttt{bridge: Client} — transport to the browser controller
  \item \texttt{default\_timeout\_s: int} — call timeout
  \item \texttt{sessions: set[SessionId]} — open sessions
\end{itemize}

\subsubsection{Environment Variables}

\begin{itemize}
  \item \texttt{BROWSER\_BRIDGE\_URL} — endpoint for the controller
\end{itemize}

\subsubsection{Assumptions}

Controller endpoint is reachable and authenticated.

\subsubsection{Access Routine Semantics}

\noindent \texttt{BrowserController}(\texttt{bridge}, \texttt{timeout\_s}):
\begin{itemize}
\item transition: set \texttt{bridge}, \texttt{default\_timeout\_s}, clear \texttt{sessions}
\item output: initialized instance
\item exception: -
\end{itemize}

\noindent \texttt{send}(\texttt{req}):
\begin{itemize}
\item transition: none
\item output: \texttt{BackendResp} from controller
\item exception: \texttt{TimeoutError}, \texttt{TransportError} on failures
\end{itemize}

\noindent \texttt{get\_status}(\texttt{cmd\_id}):
\begin{itemize}
\item transition: none
\item output: current \texttt{ExecStatus}
\item exception: \texttt{NotFoundError} if unknown
\end{itemize}

\noindent \texttt{cancel}(\texttt{cmd\_id}):
\begin{itemize}
\item transition: signal cancellation upstream
\item output: \texttt{true} if accepted
\item exception: \texttt{NotFoundError} if unknown
\end{itemize}

\noindent \texttt{open\_session}(\texttt{user\_id}):
\begin{itemize}
\item transition: create session; add to \texttt{sessions}
\item output: \texttt{SessionId}
\item exception: \texttt{TransportError} if controller rejects
\end{itemize}

\noindent \texttt{close\_session}(\texttt{session\_id}):
\begin{itemize}
\item transition: close remotely; remove from \texttt{sessions}
\item output: \texttt{true} on success; otherwise \texttt{false}
\item exception: \texttt{TransportError} on failure
\end{itemize}

\subsubsection{Local Functions}

\begin{itemize}
  \item \texttt{with\_timeout(call)} — wrap a bridge call with timeout
  \item \texttt{normalize(resp)} — normalize controller response
\end{itemize}
\section{MIS of Session Manager} \label{M10}

\subsection{Module}

Manages ongoing user sessions and their states: start, stop, track activity, and associate commands with sessions.

\subsection{Uses}

Persistent store or cache; clock utilities; audit logger.

\subsection{Syntax}

\subsubsection{Exported Constants}

None.

\subsubsection{Exported Access Programs}

\begin{center}
\begin{tabular}{p{4cm} p{4cm} p{3cm} p{4cm}}
\hline
\textbf{Name} & \textbf{In} & \textbf{Out} & \textbf{Exceptions} \\
\hline
\texttt{SessionManager} & \texttt{store: Store}, \texttt{ttl\_s: int} & self & - \\
\texttt{start} & \texttt{user\_id: UUID} & \texttt{SessionId} & \texttt{StoreError} \\
\texttt{stop} & \texttt{session\_id: SessionId} & \texttt{bool} & \texttt{StoreError} \\
\texttt{get} & \texttt{session\_id: SessionId} & \texttt{SessionState} & \texttt{NotFoundError} \\
\texttt{attach\_command} & \texttt{session\_id: SessionId}, \texttt{cmd\_id: UUID} & \texttt{bool} & \texttt{NotFoundError} \\
\texttt{set\_state} & \texttt{session\_id: SessionId}, \texttt{state: SessionState} & \texttt{bool} & \texttt{StoreError} \\
\hline
\end{tabular}
\end{center}

\subsection{Semantics}

\subsubsection{State Variables}

\begin{itemize}
  \item \texttt{store: Store} — persistence for sessions
  \item \texttt{ttl\_s: int} — idle expiry threshold
  \item \texttt{index: map[SessionId] to SessionState} — in-memory cache
\end{itemize}

\subsubsection{Environment Variables}

\begin{itemize}
  \item \texttt{SESSION\_TTL\_S} — default idle timeout
\end{itemize}

\subsubsection{Assumptions}

Store operations are atomic per session key.

\subsubsection{Access Routine Semantics}

\noindent \texttt{SessionManager}(\texttt{store}, \texttt{ttl\_s}):
\begin{itemize}
\item transition: set fields; warm cache from store if available
\item output: initialized instance
\item exception: -
\end{itemize}

\noindent \texttt{start}(\texttt{user\_id}):
\begin{itemize}
\item transition: create session; write to store; cache in \texttt{index}
\item output: \texttt{SessionId}
\item exception: \texttt{StoreError} on failure
\end{itemize}

\noindent \texttt{stop}(\texttt{session\_id}):
\begin{itemize}
\item transition: mark closed; evict from \texttt{index}; update store
\item output: \texttt{true} on success; otherwise \texttt{false}
\item exception: \texttt{StoreError} on failure
\end{itemize}

\noindent \texttt{get}(\texttt{session\_id}):
\begin{itemize}
\item transition: none
\item output: current \texttt{SessionState}
\item exception: \texttt{NotFoundError} if unknown
\end{itemize}

\noindent \texttt{attach\_command}(\texttt{session\_id}, \texttt{cmd\_id}):
\begin{itemize}
\item transition: append command reference to session state
\item output: \texttt{true} if attached
\item exception: \texttt{NotFoundError} if session unknown
\end{itemize}

\noindent \texttt{set\_state}(\texttt{session\_id}, \texttt{state}):
\begin{itemize}
\item transition: update state in cache and store
\item output: \texttt{true} if updated
\item exception: \texttt{StoreError} on write failure
\end{itemize}

\subsubsection{Local Functions}

\begin{itemize}
  \item \texttt{now()} — current timestamp
  \item \texttt{expired(state)} — checks idle expiry
\end{itemize}
\section{MIS of Error Handling \& Recovery Module} \label{M11}

\subsection{Module}

Classifies errors, applies retry and backoff policies, and coordinates recovery or compensation actions to return the system to a consistent state.

\subsection{Uses}

Policy store; timer/backoff utilities; audit logger; command execution layer; browser controller.

\subsection{Syntax}

\subsubsection{Exported Constants}

None.

\subsubsection{Exported Access Programs}

\begin{center}
\begin{tabular}{p{4cm} p{4cm} p{3cm} p{4cm}}
\hline
\textbf{Name} & \textbf{In} & \textbf{Out} & \textbf{Exceptions} \\
\hline
\texttt{ErrorHandler} & \texttt{policy: Policy} & self & - \\
\texttt{handle} & \texttt{err: ExecError}, \texttt{ctx: ExecCtx} & \texttt{ActionResult} & - \\
\texttt{retry} & \texttt{cmd\_id: UUID} & \texttt{bool} & \texttt{PolicyError} \\
\texttt{compensate} & \texttt{cmd\_id: UUID} & \texttt{bool} & \texttt{ExecError} \\
\texttt{classify} & \texttt{err: ExecError} & \texttt{ErrorClass} & - \\
\texttt{record} & \texttt{err: ExecError}, \texttt{ctx: ExecCtx} & \texttt{UUID} & - \\
\hline
\end{tabular}
\end{center}

\subsection{Semantics}

\subsubsection{State Variables}

\begin{itemize}
  \item \texttt{policy: Policy} — retry and compensation rules
  \item \texttt{history: map[UUID] to ErrorEvent} — recent errors
  \item \texttt{backoff: map[UUID] to int} — current backoff in seconds
\end{itemize}

\subsubsection{Environment Variables}

\begin{itemize}
  \item \texttt{MAX\_RETRIES} — hard cap for attempts
  \item \texttt{BASE\_BACKOFF\_S} — initial delay
\end{itemize}

\subsubsection{Assumptions}

Compensation actions are idempotent or guarded.

\subsubsection{Access Routine Semantics}

\noindent \texttt{ErrorHandler}(\texttt{policy}):
\begin{itemize}
\item transition: set \texttt{policy}; clear \texttt{history}, \texttt{backoff}
\item output: initialized instance
\item exception: -
\end{itemize}

\noindent \texttt{handle}(\texttt{err}, \texttt{ctx}):
\begin{itemize}
\item transition: classify; decide action (retry, compensate, fail); update \texttt{history}, \texttt{backoff}
\item output: \texttt{ActionResult} describing action taken
\item exception: -
\end{itemize}

\noindent \texttt{retry}(\texttt{cmd\_id}):
\begin{itemize}
\item transition: schedule retry with policy backoff
\item output: \texttt{true} if scheduled
\item exception: \texttt{PolicyError} if retries exhausted
\end{itemize}

\noindent \texttt{compensate}(\texttt{cmd\_id}):
\begin{itemize}
\item transition: execute compensation; update state
\item output: \texttt{true} on success; otherwise \texttt{false}
\item exception: \texttt{ExecError} if compensation fails
\end{itemize}

\noindent \texttt{classify}(\texttt{err}):
\begin{itemize}
\item transition: none
\item output: \texttt{ErrorClass} such as transient, permanent, or user
\item exception: -
\end{itemize}

\noindent \texttt{record}(\texttt{err}, \texttt{ctx}):
\begin{itemize}
\item transition: persist error event; update \texttt{history}
\item output: event identifier \texttt{UUID}
\item exception: -
\end{itemize}

\subsubsection{Local Functions}

\begin{itemize}
  \item \texttt{calc\_backoff(cmd\_id)} — compute next delay
  \item \texttt{is\_idempotent(cmd\_id)} — check compensation safety
\end{itemize}



\bibliographystyle {plainnat}
\bibliography {../../../refs/References}

\newpage

\section{Appendix} \label{Appendix}

\wss{Extra information if required}

\newpage{}

\section*{Appendix --- Reflection}

\wss{Not required for CAS 741 projects}

The information in this section will be used to evaluate the team members on the
graduate attribute of Problem Analysis and Design.

\input{../../Reflection.tex}

\begin{enumerate}
  \item What went well while writing this deliverable? 
  \item What pain points did you experience during this deliverable, and how
    did you resolve them?
  \item Which of your design decisions stemmed from speaking to your client(s)
  or a proxy (e.g. your peers, stakeholders, potential users)? For those that
  were not, why, and where did they come from?
  \item While creating the design doc, what parts of your other documents (e.g.
  requirements, hazard analysis, etc), it any, needed to be changed, and why?
  \item What are the limitations of your solution?  Put another way, given
  unlimited resources, what could you do to make the project better? (LO\_ProbSolutions)
  \item Give a brief overview of other design solutions you considered.  What
  are the benefits and tradeoffs of those other designs compared with the chosen
  design?  From all the potential options, why did you select the documented design?
  (LO\_Explores)
\end{enumerate}


\end{document}